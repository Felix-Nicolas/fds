% !TEX root = FDS_Validation_Guide.tex

\chapter{HGL Temperature and Depth}

\label{HGL:Chapter}

FDS, like any CFD-based fire model, does not perform a direct calculation of the HGL temperature or height. These are constructs unique to two-zone models. Nevertheless, FDS does make predictions of gas temperature at the same locations as the thermocouples in the experiments, and these values can be reduced in the same manner as the experimental measurements to produce an ``average'' HGL temperature and height.  Regardless of the validity of the reduction method, the FDS predictions of the HGL temperature and height ought to be representative of the accuracy of its predictions of the individual thermocouple measurements that are used in the HGL reduction. The temperature measurements from the experiments reported in this chapter are used to compute an HGL temperature and height with which to compare to FDS.  The same layer reduction method, described in the next section, is used for all the data presented in this chapter.


\section{HGL Reduction Method}
\label{HGL_Reduction}

Fire protection engineers often need to estimate the location of the interface between the hot, smoke-laden upper layer and the cooler lower layer in a burning compartment.  Relatively simple fire models, often referred to as {\em two-zone models}, compute this quantity directly, along with the average temperature of the upper and lower layers.  In a CFD-based fire model like FDS, there are not two distinct zones, but rather a continuous profile of temperature. Nevertheless, there are methods that have been developed to estimate layer height and average temperatures from a continuous vertical profile of temperature. One such method~\cite{Janssens:JFS1992} is as follows: Consider a continuous function $T(z)$ defining temperature $T$ as a function of height above the floor $z$, where $z=0$ is the floor and $z=H$ is the ceiling. Define $T_{\rm u}$ as the upper layer temperature, $T_{\rm \ell}$ as the lower layer temperature, and $z_{\rm int}$ as the interface height. Compute the quantities:
\begin{eqnarray*} (H-z_{\rm int})\; T_{\rm u} + z_{\rm int} \; T_{\rm \ell} = \int_0^H \; T(z) \; dz &=& I_1 \\
                  (H-z_{\rm int})\; \frac{1}{T_{\rm u}} + z_{\rm int} \; \frac{1}{T_{\rm \ell}} = \int_0^H \; \frac{1}{T(z)} \; dz &=& I_2 \end{eqnarray*}
Solve for $z_{\rm int}$:
\be
   z_{\rm int} = \frac{ T_{\rm \ell}(I_1 \, I_2 - H^2)}{I_1+I_2 \, T_{\rm \ell}^2 - 2\, T_{\rm \ell} \, H}
\ee
Let $T_{\rm \ell}$ be the temperature in the lowest mesh cell and, using Simpson's Rule, perform the numerical integration of $I_1$ and $I_2$. $T_{\rm u}$ is defined as the average upper layer temperature via
\be
   (H-z_{\rm int})\; T_{\rm u} = \int_{z_{\rm int}}^H \; T(z) \; dz
\ee
Further discussion of similar procedures can be found in Ref.~\cite{He:1}.

\newpage


\section{ATF Corridors}

The ATF Corridors experiments consisted of two corridors one on top of the other and connected by a stairwell. HGL temperature and depth reductions were carried out using three arrays of thermocouples in the lower corridor (Trees~A, B, and C) and two arrays in the upper corridor (Trees~G and H).

\begin{figure}[!ht]
\begin{tabular*}{\textwidth}{l@{\extracolsep{\fill}}r}
\includegraphics[height=2.15in]{SCRIPT_FIGURES/ATF_Corridors/ATF_Corridors_HGL_Temp_1_050_kW} &
\includegraphics[height=2.15in]{SCRIPT_FIGURES/ATF_Corridors/ATF_Corridors_HGL_Height_1_050_kW} \\
\includegraphics[height=2.15in]{SCRIPT_FIGURES/ATF_Corridors/ATF_Corridors_HGL_Temp_1_100_kW} &
\includegraphics[height=2.15in]{SCRIPT_FIGURES/ATF_Corridors/ATF_Corridors_HGL_Height_1_100_kW} \\
\includegraphics[height=2.15in]{SCRIPT_FIGURES/ATF_Corridors/ATF_Corridors_HGL_Temp_1_240_kW} &
\includegraphics[height=2.15in]{SCRIPT_FIGURES/ATF_Corridors/ATF_Corridors_HGL_Height_1_240_kW}
\end{tabular*}
\caption[ATF Corridors, HGL temperature and height, first floor, 50~kW, 100~kW, 240~kW]
{ATF Corridors, HGL temperature and height, first floor, 50~kW, 100~kW, 240~kW.}
\label{ATF_Corridors_HGL_1}
\end{figure}

\newpage

\begin{figure}[p]
\begin{tabular*}{\textwidth}{l@{\extracolsep{\fill}}r}
\includegraphics[height=2.15in]{SCRIPT_FIGURES/ATF_Corridors/ATF_Corridors_HGL_Temp_1_250_kW} &
\includegraphics[height=2.15in]{SCRIPT_FIGURES/ATF_Corridors/ATF_Corridors_HGL_Height_1_250_kW} \\
\includegraphics[height=2.15in]{SCRIPT_FIGURES/ATF_Corridors/ATF_Corridors_HGL_Temp_1_500_kW} &
\includegraphics[height=2.15in]{SCRIPT_FIGURES/ATF_Corridors/ATF_Corridors_HGL_Height_1_500_kW} \\
\includegraphics[height=2.15in]{SCRIPT_FIGURES/ATF_Corridors/ATF_Corridors_HGL_Temp_1_Mix_kW} &
\includegraphics[height=2.15in]{SCRIPT_FIGURES/ATF_Corridors/ATF_Corridors_HGL_Height_1_Mix_kW}
\end{tabular*}
\caption[ATF Corridors, HGL temperature and height, first floor, 250~kW, 500~kW, mixed]
{ATF Corridors, HGL temperature and height, first floor, 250~kW, 500~kW, mixed.}
\label{ATF_Corridors_HGL_2}
\end{figure}

\begin{figure}[p]
\begin{tabular*}{\textwidth}{l@{\extracolsep{\fill}}r}
\includegraphics[height=2.15in]{SCRIPT_FIGURES/ATF_Corridors/ATF_Corridors_HGL_Temp_2_050_kW} &
\includegraphics[height=2.15in]{SCRIPT_FIGURES/ATF_Corridors/ATF_Corridors_HGL_Height_2_050_kW} \\
\includegraphics[height=2.15in]{SCRIPT_FIGURES/ATF_Corridors/ATF_Corridors_HGL_Temp_2_100_kW} &
\includegraphics[height=2.15in]{SCRIPT_FIGURES/ATF_Corridors/ATF_Corridors_HGL_Height_2_100_kW} \\
\includegraphics[height=2.15in]{SCRIPT_FIGURES/ATF_Corridors/ATF_Corridors_HGL_Temp_2_240_kW} &
\includegraphics[height=2.15in]{SCRIPT_FIGURES/ATF_Corridors/ATF_Corridors_HGL_Height_2_240_kW}
\end{tabular*}
\caption[ATF Corridors, HGL temperature and height, second floor, 50~kW, 100~kW, 240~kW]
{ATF Corridors, HGL temperature and height, second floor, 50~kW, 100~kW, 240~kW.}
\label{ATF_Corridors_HGL_3}
\end{figure}

\begin{figure}[p]
\begin{tabular*}{\textwidth}{l@{\extracolsep{\fill}}r}
\includegraphics[height=2.15in]{SCRIPT_FIGURES/ATF_Corridors/ATF_Corridors_HGL_Temp_2_250_kW} &
\includegraphics[height=2.15in]{SCRIPT_FIGURES/ATF_Corridors/ATF_Corridors_HGL_Height_2_250_kW} \\
\includegraphics[height=2.15in]{SCRIPT_FIGURES/ATF_Corridors/ATF_Corridors_HGL_Temp_2_500_kW} &
\includegraphics[height=2.15in]{SCRIPT_FIGURES/ATF_Corridors/ATF_Corridors_HGL_Height_2_500_kW} \\
\includegraphics[height=2.15in]{SCRIPT_FIGURES/ATF_Corridors/ATF_Corridors_HGL_Temp_2_Mix_kW} &
\includegraphics[height=2.15in]{SCRIPT_FIGURES/ATF_Corridors/ATF_Corridors_HGL_Height_2_Mix_kW}
\end{tabular*}
\caption[ATF Corridors, HGL temperature and height, second floor, 250~kW, 500~kW, mixed]
{ATF Corridors, HGL temperature and height, second floor, 250~kW, 500~kW, mixed.}
\label{ATF_Corridors_HGL_4}
\end{figure}

\clearpage


\section{CSTB Tunnel}

The CSTB Tunnel experiments include thermocouple measurements at various locations in a small-scale tunnel equipped with a water mist system. Two experiments (Tests~2 and 27) are simulated; the former with no mist activation and the latter with activation after 5~min. The tunnel is approximately 43~m long and the fire is located 17.5~m from the upstream opening.

On the following pages, thermocouple measurements at downwind distances of 4~m, 8~m, 12~m, and 24~m are compared to the predicted values. For a given label, say T+2411, the T denotes Temperature, the +24 means 24~m downwind of the fire, and the final two digits denote the particular thermocouple within an array at that location. The chosen TCs are located along the vertical centerline.

\begin{figure}[!ht]
\centering
\includegraphics[height=2.15in]{SCRIPT_FIGURES/CSTB_Tunnel/CSTB_Tunnel_Test_2_Temperature_Section+4}  \\
\includegraphics[height=2.15in]{SCRIPT_FIGURES/CSTB_Tunnel/CSTB_Tunnel_Test_2_Temperature_Section+8}  \\
\includegraphics[height=2.15in]{SCRIPT_FIGURES/CSTB_Tunnel/CSTB_Tunnel_Test_2_Temperature_Section+12} \\
\includegraphics[height=2.15in]{SCRIPT_FIGURES/CSTB_Tunnel/CSTB_Tunnel_Test_2_Temperature_Section+24}
\caption[CSTB Tunnel, Test 2, temperatures at 4, 8, 12 and 24 m downstream of fire]
{CSTB Tunnel, Test 2, temperatures at 4, 8, 12 and 24 m downstream of fire.}
\label{CSTB_Tunnel_1}
\end{figure}

\begin{figure}[!ht]
\centering
\includegraphics[height=2.15in]{SCRIPT_FIGURES/CSTB_Tunnel/CSTB_Tunnel_Test_27_Temperature_Section+4}  \\
\includegraphics[height=2.15in]{SCRIPT_FIGURES/CSTB_Tunnel/CSTB_Tunnel_Test_27_Temperature_Section+8}  \\
\includegraphics[height=2.15in]{SCRIPT_FIGURES/CSTB_Tunnel/CSTB_Tunnel_Test_27_Temperature_Section+12} \\
\includegraphics[height=2.15in]{SCRIPT_FIGURES/CSTB_Tunnel/CSTB_Tunnel_Test_27_Temperature_Section+24}
\caption[CSTB Tunnel, Test 27, temperatures at 4, 8, 12 and 24 m downstream of fire]
{CSTB Tunnel, Test 27, temperatures at 4, 8, 12 and 24 m downstream of fire.}
\label{CSTB_Tunnel_2}
\end{figure}


\clearpage

\section{DelCo Trainers}

The DelCo Trainer experiments were conducted in two different structures. Tests 2-6 were conducted in a single level structure consisting of three rooms. Rooms 1 and 3 had two thermocouple trees and Room 2 had one. Tests~22-25 were conducted in a two level structure. Floors~1 and 2 each had three thermocouple arrays. See Sec.~\ref{DelCo_Description} for their exact locations.

\newpage


\begin{figure}[p]
\begin{tabular*}{\textwidth}{l@{\extracolsep{\fill}}r}
\includegraphics[height=2.15in]{SCRIPT_FIGURES/DelCo_Trainers/Test_02_HGL_Temp_Room_1} &
\includegraphics[height=2.15in]{SCRIPT_FIGURES/DelCo_Trainers/Test_02_HGL_Height_Room_1} \\
\includegraphics[height=2.15in]{SCRIPT_FIGURES/DelCo_Trainers/Test_02_HGL_Temp_Room_2} &
\includegraphics[height=2.15in]{SCRIPT_FIGURES/DelCo_Trainers/Test_02_HGL_Height_Room_2} \\
\includegraphics[height=2.15in]{SCRIPT_FIGURES/DelCo_Trainers/Test_02_HGL_Temp_Room_3} &
\includegraphics[height=2.15in]{SCRIPT_FIGURES/DelCo_Trainers/Test_02_HGL_Height_Room_3}
\end{tabular*}
\caption[DelCo Trainers, HGL Temperature and Height, Test 2]
{DelCo Trainers, HGL Temperature and Height, Test 2.}
\label{DelCo_HGL_2}
\end{figure}

\begin{figure}[p]
\begin{tabular*}{\textwidth}{l@{\extracolsep{\fill}}r}
\includegraphics[height=2.15in]{SCRIPT_FIGURES/DelCo_Trainers/Test_03_HGL_Temp_Room_1} &
\includegraphics[height=2.15in]{SCRIPT_FIGURES/DelCo_Trainers/Test_03_HGL_Height_Room_1} \\
\includegraphics[height=2.15in]{SCRIPT_FIGURES/DelCo_Trainers/Test_03_HGL_Temp_Room_2} &
\includegraphics[height=2.15in]{SCRIPT_FIGURES/DelCo_Trainers/Test_03_HGL_Height_Room_2} \\
\includegraphics[height=2.15in]{SCRIPT_FIGURES/DelCo_Trainers/Test_03_HGL_Temp_Room_3} &
\includegraphics[height=2.15in]{SCRIPT_FIGURES/DelCo_Trainers/Test_03_HGL_Height_Room_3}
\end{tabular*}
\caption[DelCo Trainers, HGL Temperature and Height, Test 3]
{DelCo Trainers, HGL Temperature and Height, Test 3.}
\label{DelCo_HGL_3}
\end{figure}

\begin{figure}[p]
\begin{tabular*}{\textwidth}{l@{\extracolsep{\fill}}r}
\includegraphics[height=2.15in]{SCRIPT_FIGURES/DelCo_Trainers/Test_04_HGL_Temp_Room_1} &
\includegraphics[height=2.15in]{SCRIPT_FIGURES/DelCo_Trainers/Test_04_HGL_Height_Room_1} \\
\includegraphics[height=2.15in]{SCRIPT_FIGURES/DelCo_Trainers/Test_04_HGL_Temp_Room_2} &
\includegraphics[height=2.15in]{SCRIPT_FIGURES/DelCo_Trainers/Test_04_HGL_Height_Room_2} \\
\includegraphics[height=2.15in]{SCRIPT_FIGURES/DelCo_Trainers/Test_04_HGL_Temp_Room_3} &
\includegraphics[height=2.15in]{SCRIPT_FIGURES/DelCo_Trainers/Test_04_HGL_Height_Room_3}
\end{tabular*}
\caption[DelCo Trainers, HGL Temperature and Height, Test 4]
{DelCo Trainers, HGL Temperature and Height, Test 4.}
\label{DelCo_HGL_4}
\end{figure}

\begin{figure}[p]
\begin{tabular*}{\textwidth}{l@{\extracolsep{\fill}}r}
\includegraphics[height=2.15in]{SCRIPT_FIGURES/DelCo_Trainers/Test_05_HGL_Temp_Room_1} &
\includegraphics[height=2.15in]{SCRIPT_FIGURES/DelCo_Trainers/Test_05_HGL_Height_Room_1} \\
\includegraphics[height=2.15in]{SCRIPT_FIGURES/DelCo_Trainers/Test_05_HGL_Temp_Room_2} &
\includegraphics[height=2.15in]{SCRIPT_FIGURES/DelCo_Trainers/Test_05_HGL_Height_Room_2} \\
\includegraphics[height=2.15in]{SCRIPT_FIGURES/DelCo_Trainers/Test_05_HGL_Temp_Room_3} &
\includegraphics[height=2.15in]{SCRIPT_FIGURES/DelCo_Trainers/Test_05_HGL_Height_Room_3}
\end{tabular*}
\caption[DelCo Trainers, HGL Temperature and Height, Test 5]
{DelCo Trainers, HGL Temperature and Height, Test 5.}
\label{DelCo_HGL_5}
\end{figure}

\begin{figure}[p]
\begin{tabular*}{\textwidth}{l@{\extracolsep{\fill}}r}
\includegraphics[height=2.15in]{SCRIPT_FIGURES/DelCo_Trainers/Test_06_HGL_Temp_Room_1} &
\includegraphics[height=2.15in]{SCRIPT_FIGURES/DelCo_Trainers/Test_06_HGL_Height_Room_1} \\
\includegraphics[height=2.15in]{SCRIPT_FIGURES/DelCo_Trainers/Test_06_HGL_Temp_Room_2} &
\includegraphics[height=2.15in]{SCRIPT_FIGURES/DelCo_Trainers/Test_06_HGL_Height_Room_2} \\
\includegraphics[height=2.15in]{SCRIPT_FIGURES/DelCo_Trainers/Test_06_HGL_Temp_Room_3} &
\includegraphics[height=2.15in]{SCRIPT_FIGURES/DelCo_Trainers/Test_06_HGL_Height_Room_3}
\end{tabular*}
\caption[DelCo Trainers, HGL Temperature and Height, Test 6]
{DelCo Trainers, HGL Temperature and Height, Test 6.}
\label{DelCo_HGL_6}
\end{figure}

\begin{figure}[p]
\begin{tabular*}{\textwidth}{l@{\extracolsep{\fill}}r}
\includegraphics[height=2.15in]{SCRIPT_FIGURES/DelCo_Trainers/Test_22_HGL_Temp_Floor_1} &
\includegraphics[height=2.15in]{SCRIPT_FIGURES/DelCo_Trainers/Test_22_HGL_Height_Floor_1} \\
\includegraphics[height=2.15in]{SCRIPT_FIGURES/DelCo_Trainers/Test_22_HGL_Temp_Floor_2} &
\includegraphics[height=2.15in]{SCRIPT_FIGURES/DelCo_Trainers/Test_22_HGL_Height_Floor_2} \\
\includegraphics[height=2.15in]{SCRIPT_FIGURES/DelCo_Trainers/Test_23_HGL_Temp_Floor_1} &
\includegraphics[height=2.15in]{SCRIPT_FIGURES/DelCo_Trainers/Test_23_HGL_Height_Floor_1} \\
\includegraphics[height=2.15in]{SCRIPT_FIGURES/DelCo_Trainers/Test_23_HGL_Temp_Floor_2} &
\includegraphics[height=2.15in]{SCRIPT_FIGURES/DelCo_Trainers/Test_23_HGL_Height_Floor_2}
\end{tabular*}
\caption[DelCo Trainers, HGL Temperature, Tests 22-23]
{DelCo Trainers, HGL Temperature, Tests 22-23.}
\label{DelCo_HGL_22-23}
\end{figure}

\begin{figure}[p]
\begin{tabular*}{\textwidth}{l@{\extracolsep{\fill}}r}
\includegraphics[height=2.15in]{SCRIPT_FIGURES/DelCo_Trainers/Test_24_HGL_Temp_Floor_1} &
\includegraphics[height=2.15in]{SCRIPT_FIGURES/DelCo_Trainers/Test_24_HGL_Height_Floor_1} \\
\includegraphics[height=2.15in]{SCRIPT_FIGURES/DelCo_Trainers/Test_24_HGL_Temp_Floor_2} &
\includegraphics[height=2.15in]{SCRIPT_FIGURES/DelCo_Trainers/Test_24_HGL_Height_Floor_2} \\
\includegraphics[height=2.15in]{SCRIPT_FIGURES/DelCo_Trainers/Test_25_HGL_Temp_Floor_1} &
\includegraphics[height=2.15in]{SCRIPT_FIGURES/DelCo_Trainers/Test_25_HGL_Height_Floor_1} \\
\includegraphics[height=2.15in]{SCRIPT_FIGURES/DelCo_Trainers/Test_25_HGL_Temp_Floor_2} &
\includegraphics[height=2.15in]{SCRIPT_FIGURES/DelCo_Trainers/Test_25_HGL_Height_Floor_2}
\end{tabular*}
\caption[DelCo Trainers, HGL Temperature, Tests 24-25]
{DelCo Trainers, HGL Temperature, Tests 24-25.}
\label{DelCo_HGL_24-25}
\end{figure}



\clearpage

\section{FM/SNL Test Series}

Nineteen tests from the FM/SNL test series were selected for comparison. The HGL temperature and height are calculated using the standard method. The thermocouple arrays that were located in Sectors~1, 2 and 3 are averaged (with an equal weighting for each) for all tests except Tests~21 and 22. For these tests, only Sectors~1 and 3 are used, as Sector~2 falls within the smoke plume. Also, for all but the gas burner experiments, the time history of the HRR is estimated. Only the peak HRR is reported.


\begin{figure}[!h]
\begin{tabular*}{\textwidth}{l@{\extracolsep{\fill}}r}
\includegraphics[height=2.15in]{SCRIPT_FIGURES/FM_SNL/FM_SNL_01_HGL_Temp} &
\includegraphics[height=2.15in]{SCRIPT_FIGURES/FM_SNL/FM_SNL_01_HGL_Height} \\
\includegraphics[height=2.15in]{SCRIPT_FIGURES/FM_SNL/FM_SNL_02_HGL_Temp} &
\includegraphics[height=2.15in]{SCRIPT_FIGURES/FM_SNL/FM_SNL_02_HGL_Height} \\
\includegraphics[height=2.15in]{SCRIPT_FIGURES/FM_SNL/FM_SNL_03_HGL_Temp} &
\includegraphics[height=2.15in]{SCRIPT_FIGURES/FM_SNL/FM_SNL_03_HGL_Height}
\end{tabular*}
\caption[FM/SNL experiments, HGL temperature and height, Tests 1, 2, 3]
{FM/SNL experiments, HGL temperature and height, Tests 1, 2, 3.}
\label{FM_SNL_HGL_1}
\end{figure}

\newpage

\begin{figure}[p]
\begin{tabular*}{\textwidth}{l@{\extracolsep{\fill}}r}
\includegraphics[height=2.15in]{SCRIPT_FIGURES/FM_SNL/FM_SNL_04_HGL_Temp} &
\includegraphics[height=2.15in]{SCRIPT_FIGURES/FM_SNL/FM_SNL_04_HGL_Height} \\
\includegraphics[height=2.15in]{SCRIPT_FIGURES/FM_SNL/FM_SNL_05_HGL_Temp} &
\includegraphics[height=2.15in]{SCRIPT_FIGURES/FM_SNL/FM_SNL_05_HGL_Height} \\
\includegraphics[height=2.15in]{SCRIPT_FIGURES/FM_SNL/FM_SNL_06_HGL_Temp} &
\includegraphics[height=2.15in]{SCRIPT_FIGURES/FM_SNL/FM_SNL_06_HGL_Height} \\
\includegraphics[height=2.15in]{SCRIPT_FIGURES/FM_SNL/FM_SNL_07_HGL_Temp} &
\includegraphics[height=2.15in]{SCRIPT_FIGURES/FM_SNL/FM_SNL_07_HGL_Height}
\end{tabular*}
\caption[FM/SNL experiments, HGL temperature and height, Tests 4, 5, 6, 7]
{FM/SNL experiments, HGL temperature and height, Tests 4, 5, 6, 7.}
\label{FM_SNL_HGL_2}
\end{figure}

\begin{figure}[p]
\begin{tabular*}{\textwidth}{l@{\extracolsep{\fill}}r}
\includegraphics[height=2.15in]{SCRIPT_FIGURES/FM_SNL/FM_SNL_08_HGL_Temp} &
\includegraphics[height=2.15in]{SCRIPT_FIGURES/FM_SNL/FM_SNL_08_HGL_Height} \\
\includegraphics[height=2.15in]{SCRIPT_FIGURES/FM_SNL/FM_SNL_09_HGL_Temp} &
\includegraphics[height=2.15in]{SCRIPT_FIGURES/FM_SNL/FM_SNL_09_HGL_Height} \\
\includegraphics[height=2.15in]{SCRIPT_FIGURES/FM_SNL/FM_SNL_10_HGL_Temp} &
\includegraphics[height=2.15in]{SCRIPT_FIGURES/FM_SNL/FM_SNL_10_HGL_Height} \\
\includegraphics[height=2.15in]{SCRIPT_FIGURES/FM_SNL/FM_SNL_11_HGL_Temp} &
\includegraphics[height=2.15in]{SCRIPT_FIGURES/FM_SNL/FM_SNL_11_HGL_Height}
\end{tabular*}
\caption[FM/SNL experiments, HGL temperature and height, Tests 8, 9, 10, 11]
{FM/SNL experiments, HGL temperature and height, Tests 8, 9, 10, 11.}
\label{FM_SNL_HGL_3}
\end{figure}

\begin{figure}[p]
\begin{tabular*}{\textwidth}{l@{\extracolsep{\fill}}r}
\includegraphics[height=2.15in]{SCRIPT_FIGURES/FM_SNL/FM_SNL_12_HGL_Temp} &
\includegraphics[height=2.15in]{SCRIPT_FIGURES/FM_SNL/FM_SNL_12_HGL_Height} \\
\includegraphics[height=2.15in]{SCRIPT_FIGURES/FM_SNL/FM_SNL_13_HGL_Temp} &
\includegraphics[height=2.15in]{SCRIPT_FIGURES/FM_SNL/FM_SNL_13_HGL_Height} \\
\includegraphics[height=2.15in]{SCRIPT_FIGURES/FM_SNL/FM_SNL_14_HGL_Temp} &
\includegraphics[height=2.15in]{SCRIPT_FIGURES/FM_SNL/FM_SNL_14_HGL_Height} \\
\includegraphics[height=2.15in]{SCRIPT_FIGURES/FM_SNL/FM_SNL_15_HGL_Temp} &
\includegraphics[height=2.15in]{SCRIPT_FIGURES/FM_SNL/FM_SNL_15_HGL_Height}
\end{tabular*}
\caption[FM/SNL experiments, HGL temperature and height, Tests 12, 13, 14, 15]
{FM/SNL experiments, HGL temperature and height, Tests 12, 13, 14, 15.}
\label{FM_SNL_HGL_4}
\end{figure}


\begin{figure}[p]
\begin{tabular*}{\textwidth}{l@{\extracolsep{\fill}}r}
\includegraphics[height=2.15in]{SCRIPT_FIGURES/FM_SNL/FM_SNL_16_HGL_Temp} &
\includegraphics[height=2.15in]{SCRIPT_FIGURES/FM_SNL/FM_SNL_16_HGL_Height} \\
\includegraphics[height=2.15in]{SCRIPT_FIGURES/FM_SNL/FM_SNL_17_HGL_Temp} &
\includegraphics[height=2.15in]{SCRIPT_FIGURES/FM_SNL/FM_SNL_17_HGL_Height} \\
\includegraphics[height=2.15in]{SCRIPT_FIGURES/FM_SNL/FM_SNL_21_HGL_Temp} &
\includegraphics[height=2.15in]{SCRIPT_FIGURES/FM_SNL/FM_SNL_21_HGL_Height} \\
\includegraphics[height=2.15in]{SCRIPT_FIGURES/FM_SNL/FM_SNL_22_HGL_Temp} &
\includegraphics[height=2.15in]{SCRIPT_FIGURES/FM_SNL/FM_SNL_22_HGL_Height}
\end{tabular*}
\caption[FM/SNL experiments, HGL temperature and height, Tests 16, 17, 21, 22]
{FM/SNL experiments, HGL temperature and height, Tests 16, 17, 21, 22.}
\label{FM_SNL_HGL_5}
\end{figure}

\clearpage


\clearpage

\section{JH/FRA Experiments}

Eight tests from the JH/FRA test series were selected for comparison. The HGL temperature and height are calculated using the standard method. The thermocouple arrays that were located in the north east and south east corners are averaged (with an equal weighting for each) for all tests.


\begin{figure}[!h]
\begin{tabular*}{\textwidth}{l@{\extracolsep{\fill}}r}
\includegraphics[height=2.15in]{SCRIPT_FIGURES/JH_FRA/JH_FRA_compartment_01_HGL_Temp_1} &
\includegraphics[height=2.15in]{SCRIPT_FIGURES/JH_FRA/JH_FRA_compartment_01_HGL_Height_1} \\
\includegraphics[height=2.15in]{SCRIPT_FIGURES/JH_FRA/JH_FRA_compartment_02_HGL_Temp_1} &
\includegraphics[height=2.15in]{SCRIPT_FIGURES/JH_FRA/JH_FRA_compartment_02_HGL_Height_1} \\
\end{tabular*}
\caption[JH/FRA experiments, HGL temperature and height, 1:4 scale inert lining]
{JH/FRA experiments, HGL temperature and height, 1:4 scale inert lining configuration.}
\label{JH_FRA_HGL_1}
\end{figure}

\newpage

\begin{figure}[!h]
\begin{tabular*}{\textwidth}{l@{\extracolsep{\fill}}r}
\includegraphics[height=2.15in]{SCRIPT_FIGURES/JH_FRA/JH_FRA_compartment_03_HGL_Temp_1} &
\includegraphics[height=2.15in]{SCRIPT_FIGURES/JH_FRA/JH_FRA_compartment_03_HGL_Height_1} \\
\includegraphics[height=2.15in]{SCRIPT_FIGURES/JH_FRA/JH_FRA_compartment_03A_HGL_Temp_1} &
\includegraphics[height=2.15in]{SCRIPT_FIGURES/JH_FRA/JH_FRA_compartment_03A_HGL_Height_1} \\
%\includegraphics[height=2.15in]{SCRIPT_FIGURES/JH_FRA/JH_FRA_compartment_04_HGL_Temp_1} &
%\includegraphics[height=2.15in]{SCRIPT_FIGURES/JH_FRA/JH_FRA_compartment_04_HGL_Height_1} \\
\end{tabular*}
\caption[JH/FRA experiments, HGL temperature and height, 1:4 scale combustible lining]
{JH/FRA experiments, HGL temperature and height, 1:4 scale combustible lining configuration.}
\label{JH_FRA_HGL_2}
\end{figure}

\newpage

\begin{figure}[!h]
\begin{tabular*}{\textwidth}{l@{\extracolsep{\fill}}r}
\includegraphics[height=2.15in]{SCRIPT_FIGURES/JH_FRA/JH_FRA_compartment_11_HGL_Temp_1} &
\includegraphics[height=2.15in]{SCRIPT_FIGURES/JH_FRA/JH_FRA_compartment_11_HGL_Height_1} \\
\includegraphics[height=2.15in]{SCRIPT_FIGURES/JH_FRA/JH_FRA_compartment_12_HGL_Temp_1} &
\includegraphics[height=2.15in]{SCRIPT_FIGURES/JH_FRA/JH_FRA_compartment_12_HGL_Height_1} \\
\end{tabular*}
\caption[JH/FRA experiments, HGL temperature and height, 1:2 scale inert lining]
{JH/FRA experiments, HGL temperature and height, 1:2 scale inert lining configuration.}
\label{JH_FRA_HGL_3}
\end{figure}

\newpage

\begin{figure}[!h]
\begin{tabular*}{\textwidth}{l@{\extracolsep{\fill}}r}
\includegraphics[height=2.15in]{SCRIPT_FIGURES/JH_FRA/JH_FRA_compartment_13_HGL_Temp_1} &
\includegraphics[height=2.15in]{SCRIPT_FIGURES/JH_FRA/JH_FRA_compartment_13_HGL_Height_1} \\
%\includegraphics[height=2.15in]{SCRIPT_FIGURES/JH_FRA/JH_FRA_compartment_14_HGL_Temp_1} &
%\includegraphics[height=2.15in]{SCRIPT_FIGURES/JH_FRA/JH_FRA_compartment_14_HGL_Height_1} \\
\end{tabular*}
\caption[JH/FRA experiments, HGL temperature and height, 1:2 scale combustible lining]
{JH/FRA experiments, HGL temperature and height, 1:2 scale combustible lining configuration.}
\label{JH_FRA_HGL_4}
\end{figure}

\newpage

\begin{figure}[!h]
\begin{tabular*}{\textwidth}{l@{\extracolsep{\fill}}r}
\includegraphics[height=2.15in]{SCRIPT_FIGURES/JH_FRA/JH_FRA_compartment_21_HGL_Temp_1} &
\includegraphics[height=2.15in]{SCRIPT_FIGURES/JH_FRA/JH_FRA_compartment_21_HGL_Height_1} \\
\includegraphics[height=2.15in]{SCRIPT_FIGURES/JH_FRA/JH_FRA_compartment_22_HGL_Temp_1} &
\includegraphics[height=2.15in]{SCRIPT_FIGURES/JH_FRA/JH_FRA_compartment_22_HGL_Height_1} \\
\end{tabular*}
\caption[JH/FRA experiments, HGL temperature and height, 1:1 scale inert lining]
{JH/FRA experiments, HGL temperature and height, 1:1 scale inert lining configuration.}
\label{JH_FRA_HGL_5}
\end{figure}

\newpage

%\begin{figure}[!h]
%\begin{tabular*}{\textwidth}{l@{\extracolsep{\fill}}r}
%\includegraphics[height=2.15in]{SCRIPT_FIGURES/JH_FRA/JH_FRA_compartment_23_HGL_Temp_1} &
%\includegraphics[height=2.15in]{SCRIPT_FIGURES/JH_FRA/JH_FRA_compartment_23_HGL_Height_1} \\
%\includegraphics[height=2.15in]{SCRIPT_FIGURES/JH_FRA/JH_FRA_compartment_24_HGL_Temp_1} &
%\includegraphics[height=2.15in]{SCRIPT_FIGURES/JH_FRA/JH_FRA_compartment_24_HGL_Height_1} \\
%\end{tabular*}
%\caption[JH/FRA experiments, HGL temperature and height, 1:1 scale combustible lining]
%{JH/FRA experiments, HGL temperature and height, 1:1 scale combustible lining configuration.}
%\label{JH_FRA_HGL_6}
%\end{figure}
%
%\newpage



\section{LLNL Enclosure Series}

The figures on the following pages compare predicted and measured hot gas layer temperatures from the LLNL Enclosure experiments. Fifteen thermocouples were evenly spaced from floor to ceiling on either side of the burner. The measured temperatures were reported as averages of the lower, middle, and upper five TCs. Some of the experiments were conducted with a separated plenum space in the top one-third of the overall compartment (Tests~17-60). In these cases, the upper five TCs are a measure of the average plenum temperature.

In the figures, the black circles represent the average of the five upper-most TC measurements. The red circles represent the average of the middle five TC measurements. The corresponding colored curves represent the simulation. For the experiments involving an upper plenum, the middle five TCs are located immediately beneath the plenum and their average temperature is typically greater than that of the upper-most TCs in the plenum. Note that in a number of experiments, the fuel flow was stopped or the fire self-extinguished. The simulations last only as long as the reported measurements.

Details on the experiments and modeling can be found in Sec.~\ref{LLNL_Enclosure_Description}.

\newpage

\begin{figure}[p]
\begin{tabular*}{\textwidth}{l@{\extracolsep{\fill}}r}
\includegraphics[height=2.15in]{SCRIPT_FIGURES/LLNL_Enclosure/LLNL_01_Temp} &
\includegraphics[height=2.15in]{SCRIPT_FIGURES/LLNL_Enclosure/LLNL_02_Temp} \\
\includegraphics[height=2.15in]{SCRIPT_FIGURES/LLNL_Enclosure/LLNL_03_Temp} &
\includegraphics[height=2.15in]{SCRIPT_FIGURES/LLNL_Enclosure/LLNL_04_Temp} \\
\includegraphics[height=2.15in]{SCRIPT_FIGURES/LLNL_Enclosure/LLNL_05_Temp} &
\includegraphics[height=2.15in]{SCRIPT_FIGURES/LLNL_Enclosure/LLNL_06_Temp} \\
\includegraphics[height=2.15in]{SCRIPT_FIGURES/LLNL_Enclosure/LLNL_07_Temp} &
\includegraphics[height=2.15in]{SCRIPT_FIGURES/LLNL_Enclosure/LLNL_08_Temp}
\end{tabular*}
\caption[LLNL Enclosure experiments, HGL temperature, Tests 1-8]
{LLNL Enclosure experiments, HGL temperature, Tests 1-8.}
\label{LLNL_Enclosure_Temp_1}
\end{figure}

\begin{figure}[p]
\begin{tabular*}{\textwidth}{l@{\extracolsep{\fill}}r}
\includegraphics[height=2.15in]{SCRIPT_FIGURES/LLNL_Enclosure/LLNL_09_Temp} &
\includegraphics[height=2.15in]{SCRIPT_FIGURES/LLNL_Enclosure/LLNL_10_Temp} \\
\includegraphics[height=2.15in]{SCRIPT_FIGURES/LLNL_Enclosure/LLNL_11_Temp} &
\includegraphics[height=2.15in]{SCRIPT_FIGURES/LLNL_Enclosure/LLNL_12_Temp} \\
\includegraphics[height=2.15in]{SCRIPT_FIGURES/LLNL_Enclosure/LLNL_13_Temp} &
\includegraphics[height=2.15in]{SCRIPT_FIGURES/LLNL_Enclosure/LLNL_14_Temp} \\
\includegraphics[height=2.15in]{SCRIPT_FIGURES/LLNL_Enclosure/LLNL_15_Temp} &
\includegraphics[height=2.15in]{SCRIPT_FIGURES/LLNL_Enclosure/LLNL_16_Temp}
\end{tabular*}
\caption[LLNL Enclosure experiments, HGL temperature, Tests 9-16]
{LLNL Enclosure experiments, HGL temperature, Tests 9-16.}
\label{LLNL_Enclosure_Temp_2}
\end{figure}

\begin{figure}[p]
\begin{tabular*}{\textwidth}{l@{\extracolsep{\fill}}r}
\includegraphics[height=2.15in]{SCRIPT_FIGURES/LLNL_Enclosure/LLNL_17_Temp} &
\includegraphics[height=2.15in]{SCRIPT_FIGURES/LLNL_Enclosure/LLNL_18_Temp} \\
\includegraphics[height=2.15in]{SCRIPT_FIGURES/LLNL_Enclosure/LLNL_19_Temp} &
\includegraphics[height=2.15in]{SCRIPT_FIGURES/LLNL_Enclosure/LLNL_20_Temp} \\
\includegraphics[height=2.15in]{SCRIPT_FIGURES/LLNL_Enclosure/LLNL_21_Temp} &
\includegraphics[height=2.15in]{SCRIPT_FIGURES/LLNL_Enclosure/LLNL_22_Temp} \\
\includegraphics[height=2.15in]{SCRIPT_FIGURES/LLNL_Enclosure/LLNL_23_Temp} &
\includegraphics[height=2.15in]{SCRIPT_FIGURES/LLNL_Enclosure/LLNL_24_Temp}
\end{tabular*}
\caption[LLNL Enclosure experiments, HGL temperature, Tests 17-24]
{LLNL Enclosure experiments, HGL temperature, Tests 17-24.}
\label{LLNL_Enclosure_Temp_3}
\end{figure}

\begin{figure}[p]
\begin{tabular*}{\textwidth}{l@{\extracolsep{\fill}}r}
\includegraphics[height=2.15in]{SCRIPT_FIGURES/LLNL_Enclosure/LLNL_25_Temp} &
\includegraphics[height=2.15in]{SCRIPT_FIGURES/LLNL_Enclosure/LLNL_26_Temp} \\
\includegraphics[height=2.15in]{SCRIPT_FIGURES/LLNL_Enclosure/LLNL_27_Temp} &
\includegraphics[height=2.15in]{SCRIPT_FIGURES/LLNL_Enclosure/LLNL_28_Temp} \\
\includegraphics[height=2.15in]{SCRIPT_FIGURES/LLNL_Enclosure/LLNL_29_Temp} &
\includegraphics[height=2.15in]{SCRIPT_FIGURES/LLNL_Enclosure/LLNL_30_Temp} \\
\includegraphics[height=2.15in]{SCRIPT_FIGURES/LLNL_Enclosure/LLNL_31_Temp} &
\includegraphics[height=2.15in]{SCRIPT_FIGURES/LLNL_Enclosure/LLNL_32_Temp}
\end{tabular*}
\caption[LLNL Enclosure experiments, HGL temperature, Tests 25-32]
{LLNL Enclosure experiments, HGL temperature, Tests 25-32.}
\label{LLNL_Enclosure_Temp_4}
\end{figure}

\begin{figure}[p]
\begin{tabular*}{\textwidth}{l@{\extracolsep{\fill}}r}
\includegraphics[height=2.15in]{SCRIPT_FIGURES/LLNL_Enclosure/LLNL_33_Temp} &
\includegraphics[height=2.15in]{SCRIPT_FIGURES/LLNL_Enclosure/LLNL_34_Temp} \\
\includegraphics[height=2.15in]{SCRIPT_FIGURES/LLNL_Enclosure/LLNL_35_Temp} &
\includegraphics[height=2.15in]{SCRIPT_FIGURES/LLNL_Enclosure/LLNL_36_Temp} \\
\includegraphics[height=2.15in]{SCRIPT_FIGURES/LLNL_Enclosure/LLNL_37_Temp} &
\includegraphics[height=2.15in]{SCRIPT_FIGURES/LLNL_Enclosure/LLNL_38_Temp} \\
\includegraphics[height=2.15in]{SCRIPT_FIGURES/LLNL_Enclosure/LLNL_39_Temp} &
\includegraphics[height=2.15in]{SCRIPT_FIGURES/LLNL_Enclosure/LLNL_40_Temp}
\end{tabular*}
\caption[LLNL Enclosure experiments, HGL temperature, Tests 33-40]
{LLNL Enclosure experiments, HGL temperature, Tests 33-40.}
\label{LLNL_Enclosure_Temp_5}
\end{figure}

\begin{figure}[p]
\begin{tabular*}{\textwidth}{l@{\extracolsep{\fill}}r}
\includegraphics[height=2.15in]{SCRIPT_FIGURES/LLNL_Enclosure/LLNL_41_Temp} &
\includegraphics[height=2.15in]{SCRIPT_FIGURES/LLNL_Enclosure/LLNL_42_Temp} \\
\includegraphics[height=2.15in]{SCRIPT_FIGURES/LLNL_Enclosure/LLNL_43_Temp} &
\includegraphics[height=2.15in]{SCRIPT_FIGURES/LLNL_Enclosure/LLNL_44_Temp} \\
\includegraphics[height=2.15in]{SCRIPT_FIGURES/LLNL_Enclosure/LLNL_45_Temp} &
\includegraphics[height=2.15in]{SCRIPT_FIGURES/LLNL_Enclosure/LLNL_46_Temp} \\
\includegraphics[height=2.15in]{SCRIPT_FIGURES/LLNL_Enclosure/LLNL_47_Temp} &
\includegraphics[height=2.15in]{SCRIPT_FIGURES/LLNL_Enclosure/LLNL_48_Temp}
\end{tabular*}
\caption[LLNL Enclosure experiments, HGL temperature, Tests 41-48]
{LLNL Enclosure experiments, HGL temperature, Tests 41-48.}
\label{LLNL_Enclosure_Temp_6}
\end{figure}

\begin{figure}[p]
\begin{tabular*}{\textwidth}{l@{\extracolsep{\fill}}r}
\includegraphics[height=2.15in]{SCRIPT_FIGURES/LLNL_Enclosure/LLNL_49_Temp} &
\includegraphics[height=2.15in]{SCRIPT_FIGURES/LLNL_Enclosure/LLNL_50_Temp} \\
\includegraphics[height=2.15in]{SCRIPT_FIGURES/LLNL_Enclosure/LLNL_51_Temp} &
\includegraphics[height=2.15in]{SCRIPT_FIGURES/LLNL_Enclosure/LLNL_52_Temp} \\
\includegraphics[height=2.15in]{SCRIPT_FIGURES/LLNL_Enclosure/LLNL_53_Temp} &
\includegraphics[height=2.15in]{SCRIPT_FIGURES/LLNL_Enclosure/LLNL_54_Temp} \\
\includegraphics[height=2.15in]{SCRIPT_FIGURES/LLNL_Enclosure/LLNL_55_Temp} &
\includegraphics[height=2.15in]{SCRIPT_FIGURES/LLNL_Enclosure/LLNL_56_Temp}
\end{tabular*}
\caption[LLNL Enclosure experiments, HGL temperature, Tests 49-56]
{LLNL Enclosure experiments, HGL temperature, Tests 49-56.}
\label{LLNL_Enclosure_Temp_7}
\end{figure}

\begin{figure}[p]
\begin{tabular*}{\textwidth}{l@{\extracolsep{\fill}}r}
\includegraphics[height=2.15in]{SCRIPT_FIGURES/LLNL_Enclosure/LLNL_57_Temp} &
\includegraphics[height=2.15in]{SCRIPT_FIGURES/LLNL_Enclosure/LLNL_58_Temp} \\
\includegraphics[height=2.15in]{SCRIPT_FIGURES/LLNL_Enclosure/LLNL_59_Temp} &
\includegraphics[height=2.15in]{SCRIPT_FIGURES/LLNL_Enclosure/LLNL_60_Temp} \\
\includegraphics[height=2.15in]{SCRIPT_FIGURES/LLNL_Enclosure/LLNL_61_Temp} &
\includegraphics[height=2.15in]{SCRIPT_FIGURES/LLNL_Enclosure/LLNL_62_Temp} \\
\includegraphics[height=2.15in]{SCRIPT_FIGURES/LLNL_Enclosure/LLNL_63_Temp} &
\includegraphics[height=2.15in]{SCRIPT_FIGURES/LLNL_Enclosure/LLNL_64_Temp}
\end{tabular*}
\caption[LLNL Enclosure experiments, HGL temperature, Tests 57-64]
{LLNL Enclosure experiments, HGL temperature, Tests 57-64.}
\label{LLNL_Enclosure_Temp_8}
\end{figure}

\clearpage

\section{NBS Multi-Room Test Series}

This series of experiments was performed in two relatively small rooms connected by a long corridor. The fire was located in one of the rooms.  Eight vertical arrays of thermocouples were positioned throughout the test space: Tree~1 in the burn room, Tree~2 in the doorway of the burn room, Trees~3, 4, and 5 in the corridor, Tree~6 in the exit doorway to the outside at the far end of the corridor, Tree~7 in the doorway of the ``target'' room, and Tree~8 inside the target room.  Four trees have been selected for comparison with model prediction: Tree~1 in the burn room, the trees in the corridor, and Tree~8 in the target room in Test~100Z. In Tests~100A and 100O, the target room was closed. The test director reduced the layer information individually for the eight thermocouple arrays using an alternative method. These results were included in the original data sets. However, in this report the selected TC trees were reduced using the method described in Sec.~\ref{HGL_Reduction}.

\newpage

\begin{figure}[p]
\begin{tabular*}{\textwidth}{l@{\extracolsep{\fill}}r}
\includegraphics[height=2.15in]{SCRIPT_FIGURES/NBS/NBS_100A_Tree_1_HGL_Temp} &
\includegraphics[height=2.15in]{SCRIPT_FIGURES/NBS/NBS_100A_Tree_1_HGL_Height} \\
\includegraphics[height=2.15in]{SCRIPT_FIGURES/NBS/NBS_100A_Tree_3_HGL_Temp} &
\includegraphics[height=2.15in]{SCRIPT_FIGURES/NBS/NBS_100A_Tree_3_HGL_Height} \\
\includegraphics[height=2.15in]{SCRIPT_FIGURES/NBS/NBS_100A_Tree_4_HGL_Temp} &
\includegraphics[height=2.15in]{SCRIPT_FIGURES/NBS/NBS_100A_Tree_4_HGL_Height} \\
\includegraphics[height=2.15in]{SCRIPT_FIGURES/NBS/NBS_100A_Tree_5_HGL_Temp} &
\includegraphics[height=2.15in]{SCRIPT_FIGURES/NBS/NBS_100A_Tree_5_HGL_Height}
\end{tabular*}
\caption[NBS Multi-Room experiments, HGL temperature and height, Test 100A]
{NBS Multi-Room experiments, HGL temperature and height, Test 100A.}
\label{NBS_HGL_1}
\end{figure}

\begin{figure}[p]
\begin{tabular*}{\textwidth}{l@{\extracolsep{\fill}}r}
\includegraphics[height=2.15in]{SCRIPT_FIGURES/NBS/NBS_100O_Tree_1_HGL_Temp} &
\includegraphics[height=2.15in]{SCRIPT_FIGURES/NBS/NBS_100O_Tree_1_HGL_Height} \\
\includegraphics[height=2.15in]{SCRIPT_FIGURES/NBS/NBS_100O_Tree_3_HGL_Temp} &
\includegraphics[height=2.15in]{SCRIPT_FIGURES/NBS/NBS_100O_Tree_3_HGL_Height} \\
\includegraphics[height=2.15in]{SCRIPT_FIGURES/NBS/NBS_100O_Tree_4_HGL_Temp} &
\includegraphics[height=2.15in]{SCRIPT_FIGURES/NBS/NBS_100O_Tree_4_HGL_Height} \\
\includegraphics[height=2.15in]{SCRIPT_FIGURES/NBS/NBS_100O_Tree_5_HGL_Temp} &
\includegraphics[height=2.15in]{SCRIPT_FIGURES/NBS/NBS_100O_Tree_5_HGL_Height}
\end{tabular*}
\caption[NBS Multi-Room experiments, HGL temperature and height, Test 100O]
{NBS Multi-Room experiments, HGL temperature and height, Test 100O.}
\label{NBS_HGL_2}
\end{figure}

\begin{figure}[p]
\begin{tabular*}{\textwidth}{l@{\extracolsep{\fill}}r}
\includegraphics[height=2.15in]{SCRIPT_FIGURES/NBS/NBS_100Z_Tree_1_HGL_Temp} &
\includegraphics[height=2.15in]{SCRIPT_FIGURES/NBS/NBS_100Z_Tree_1_HGL_Height} \\
\includegraphics[height=2.15in]{SCRIPT_FIGURES/NBS/NBS_100Z_Tree_3_HGL_Temp} &
\includegraphics[height=2.15in]{SCRIPT_FIGURES/NBS/NBS_100Z_Tree_3_HGL_Height} \\
\includegraphics[height=2.15in]{SCRIPT_FIGURES/NBS/NBS_100Z_Tree_5_HGL_Temp} &
\includegraphics[height=2.15in]{SCRIPT_FIGURES/NBS/NBS_100Z_Tree_5_HGL_Height} \\
\includegraphics[height=2.15in]{SCRIPT_FIGURES/NBS/NBS_100Z_Tree_8_HGL_Temp} &
\includegraphics[height=2.15in]{SCRIPT_FIGURES/NBS/NBS_100Z_Tree_8_HGL_Height}
\end{tabular*}
\caption[NBS Multi-Room experiments, HGL temperature and height, Test 100Z]
{NBS Multi-Room experiments, HGL temperature and height, Test 100Z.}
\label{NBS_HGL_3}
\end{figure}

\clearpage


\section{NIST Composite Beam}

A brief description of the experiments is given in Sec.~\ref{NIST_Composite_Beam_Description}. The compartment interior dimensions are 12.4~m long, running east-west, 1.9~m wide, and 3.77~m high. Four experiments with fires were performed, labeled as Tests~2-5. Test~1 did not include a fire.

To measure the hot gas layer in the compartment, stainless steel sheathed thermocouples (Omega~TJ36-CAXL-14U-24 and TJ36-CAXL-38U-24) were mounted 0.8~m below the concrete slab, extending out of the compartment walls. Results are shown in Figs.~\ref{NIST_CB_HGL_1} and \ref{NIST_CB_HGL_2}. TCC1 was mounted 30~cm from the west wall and 46~cm from the north wall. TCC5 was mounted 30~cm from the east wall and 46~cm from the south wall. TCC2, TCC3, and TCC4 were mounted 46~cm from the north wall, and at positions -4.3~m, 0~m, and 4.3~m relative to the line of east-west symmetry, respectively (east is the positive direction). TCC6, TCC7, and TCC8 were mounted 46~cm from the south wall and 4.3~m, 0~m, and -4.3~m from the east-west line of symmetry, respectively.

Because of the symmetry of the experimental configuration, TCC1 and TCC5 are duplicates, TCC3 and TCC7 are duplicates, and TCC2, TCC4, TCC6, and TCC8 are duplicates.


\newpage

\begin{figure}[p]
\begin{tabular*}{\textwidth}{l@{\extracolsep{\fill}}r}
\includegraphics[height=2.15in]{SCRIPT_FIGURES/NIST_Composite_Beam/Test_2_TCC3-7} &
\includegraphics[height=2.15in]{SCRIPT_FIGURES/NIST_Composite_Beam/Test_3_TCC3-7} \\
\includegraphics[height=2.15in]{SCRIPT_FIGURES/NIST_Composite_Beam/Test_4_TCC3-7} &
\includegraphics[height=2.15in]{SCRIPT_FIGURES/NIST_Composite_Beam/Test_5_TCC3-7} \\
\includegraphics[height=2.15in]{SCRIPT_FIGURES/NIST_Composite_Beam/Test_2_TCC2-4-6-8} &
\includegraphics[height=2.15in]{SCRIPT_FIGURES/NIST_Composite_Beam/Test_3_TCC2-4-6-8} \\
\includegraphics[height=2.15in]{SCRIPT_FIGURES/NIST_Composite_Beam/Test_4_TCC2-4-6-8} &
\includegraphics[height=2.15in]{SCRIPT_FIGURES/NIST_Composite_Beam/Test_5_TCC2-4-6-8}
\end{tabular*}
\caption[NIST Composite Beam, mid-compartment HGL temperatures]
{NIST Composite Beam, mid-compartment HGL temperatures.}
\label{NIST_CB_HGL_1}
\end{figure}

\begin{figure}[p]
\begin{tabular*}{\textwidth}{l@{\extracolsep{\fill}}r}
\includegraphics[height=2.15in]{SCRIPT_FIGURES/NIST_Composite_Beam/Test_2_TCC1-5} &
\includegraphics[height=2.15in]{SCRIPT_FIGURES/NIST_Composite_Beam/Test_3_TCC1-5} \\
\includegraphics[height=2.15in]{SCRIPT_FIGURES/NIST_Composite_Beam/Test_4_TCC1-5} &
\includegraphics[height=2.15in]{SCRIPT_FIGURES/NIST_Composite_Beam/Test_5_TCC1-5}
\end{tabular*}
\caption[NIST Composite Beam, end-compartment HGL temperatures]
{NIST Composite Beam, end-compartment HGL temperatures.}
\label{NIST_CB_HGL_2}
\end{figure}

\clearpage

\section{NIST E119 Compartment}

A brief description of the experiments is given in Sec.~\ref{NIST_E119_Compartment_Description}. The compartment interior dimensions are 10.8~m long, running east-west, 7.0~m wide, and 3.8~m high. Three fire experiments were performed, labeled as Tests~1-3.

To measure the upper layer gas temperatures in the compartment, twelve stainless steel sheathed thermocouples (Omega~TJ36-CAXL-14U-24) were mounted 0.305~m below the ceiling, extending out of the ceiling slab. Results are shown in Figs.~\ref{NIST_E119_Compartment_HGL_1} through \ref{NIST_E119_Compartment_HGL_3}. Locations of TC1 through TC12 were shown in Fig.~\ref{NIST_E119_Compartment_Drawing_1}. Because of the symmetry of the experimental configuration, TC1 and TC4 are duplicates, TC2 and TC3 are duplicates, TC9 and TC12 are duplicates, and TC10, TC11 are duplicates.

\newpage

\begin{figure}[p]
\begin{tabular*}{\textwidth}{l@{\extracolsep{\fill}}r}
\includegraphics[height=2.15in]{SCRIPT_FIGURES/NIST_E119_Compartment/Test_1_TC1-4} &
\includegraphics[height=2.15in]{SCRIPT_FIGURES/NIST_E119_Compartment/Test_1_TC2-3} \\
\includegraphics[height=2.15in]{SCRIPT_FIGURES/NIST_E119_Compartment/Test_1_TC5} &
\includegraphics[height=2.15in]{SCRIPT_FIGURES/NIST_E119_Compartment/Test_1_TC6} \\
\includegraphics[height=2.15in]{SCRIPT_FIGURES/NIST_E119_Compartment/Test_1_TC7} &
\includegraphics[height=2.15in]{SCRIPT_FIGURES/NIST_E119_Compartment/Test_1_TC8} \\
\includegraphics[height=2.15in]{SCRIPT_FIGURES/NIST_E119_Compartment/Test_1_TC9-12} &
\includegraphics[height=2.15in]{SCRIPT_FIGURES/NIST_E119_Compartment/Test_1_TC10-11}
\end{tabular*}
\caption[NIST E119 Compartment Test 1, upper layer gas temperatures temperatures]
{NIST E119 Compartment Test 1, upper layer gas temperatures.}
\label{NIST_E119_Compartment_HGL_1}
\end{figure}

\begin{figure}[p]
\begin{tabular*}{\textwidth}{l@{\extracolsep{\fill}}r}
\includegraphics[height=2.15in]{SCRIPT_FIGURES/NIST_E119_Compartment/Test_2_TC1-4} &
\includegraphics[height=2.15in]{SCRIPT_FIGURES/NIST_E119_Compartment/Test_2_TC2-3} \\
\includegraphics[height=2.15in]{SCRIPT_FIGURES/NIST_E119_Compartment/Test_2_TC5} &
\includegraphics[height=2.15in]{SCRIPT_FIGURES/NIST_E119_Compartment/Test_2_TC6} \\
\includegraphics[height=2.15in]{SCRIPT_FIGURES/NIST_E119_Compartment/Test_2_TC7} &
\includegraphics[height=2.15in]{SCRIPT_FIGURES/NIST_E119_Compartment/Test_2_TC8} \\
\includegraphics[height=2.15in]{SCRIPT_FIGURES/NIST_E119_Compartment/Test_2_TC9-12} &
\includegraphics[height=2.15in]{SCRIPT_FIGURES/NIST_E119_Compartment/Test_2_TC10-11}
\end{tabular*}
\caption[NIST E119 Compartment Test 2, upper layer gas temperatures temperatures]
{NIST E119 Compartment Test 2, upper layer gas temperatures.}
\label{NIST_E119_Compartment_HGL_2}
\end{figure}

\begin{figure}[p]
\begin{tabular*}{\textwidth}{l@{\extracolsep{\fill}}r}
\includegraphics[height=2.15in]{SCRIPT_FIGURES/NIST_E119_Compartment/Test_3_TC1-4} &
\includegraphics[height=2.15in]{SCRIPT_FIGURES/NIST_E119_Compartment/Test_3_TC2-3} \\
\includegraphics[height=2.15in]{SCRIPT_FIGURES/NIST_E119_Compartment/Test_3_TC5} &
\includegraphics[height=2.15in]{SCRIPT_FIGURES/NIST_E119_Compartment/Test_3_TC6} \\
\includegraphics[height=2.15in]{SCRIPT_FIGURES/NIST_E119_Compartment/Test_3_TC7} &
\includegraphics[height=2.15in]{SCRIPT_FIGURES/NIST_E119_Compartment/Test_3_TC8} \\
\includegraphics[height=2.15in]{SCRIPT_FIGURES/NIST_E119_Compartment/Test_3_TC9-12} &
\includegraphics[height=2.15in]{SCRIPT_FIGURES/NIST_E119_Compartment/Test_3_TC10-11}
\end{tabular*}
\caption[NIST E119 Compartment Test 3, upper layer gas temperatures temperatures]
{NIST E119 Compartment Test 3, upper layer gas temperatures.}
\label{NIST_E119_Compartment_HGL_3}
\end{figure}

\clearpage

\section{NIST Full-Scale Enclosure (FSE), 2008}

Thermocouple arrays were suspended from the ceiling at two points along the centerline of the ISO~9705 compartment. The array in the front of the compartment was located 72~cm inside the door, and the array in the rear was 72~cm from the back wall. Each array consisted of 11 TCs positioned at heights of 3~cm, 30~cm, 60~cm, 90~cm, 105~cm, 120~cm, 135~cm, 150~cm, 180~cm, 210~cm, and 2.38~cm. The height of the compartment was 2.4~m. In the plots on the following the pages, the average HGL temperature and layer height are shown for experiments 8 through 32. The thermocouple arrays were not installed for experiments labeled ISONG3, ISOHept4, or ISOHept5.

\newpage

\begin{figure}[p]
\begin{tabular*}{\textwidth}{l@{\extracolsep{\fill}}r}
\includegraphics[height=2.15in]{SCRIPT_FIGURES/NIST_FSE_2008/ISOHept8_HGL_Temperature} &
\includegraphics[height=2.15in]{SCRIPT_FIGURES/NIST_FSE_2008/ISOHept8_HGL_Height} \\
\includegraphics[height=2.15in]{SCRIPT_FIGURES/NIST_FSE_2008/ISOHept9_HGL_Temperature} &
\includegraphics[height=2.15in]{SCRIPT_FIGURES/NIST_FSE_2008/ISOHept9_HGL_Height} \\
\includegraphics[height=2.15in]{SCRIPT_FIGURES/NIST_FSE_2008/ISONylon10_HGL_Temperature} &
\includegraphics[height=2.15in]{SCRIPT_FIGURES/NIST_FSE_2008/ISONylon10_HGL_Height} \\
\includegraphics[height=2.15in]{SCRIPT_FIGURES/NIST_FSE_2008/ISOPP11_HGL_Temperature} &
\includegraphics[height=2.15in]{SCRIPT_FIGURES/NIST_FSE_2008/ISOPP11_HGL_Height}
\end{tabular*}
\caption[NIST FSE, HGL temperature and height, Tests 8-11]
{NIST FSE, HGL temperature and height, Tests 8-11.}
\label{NIST_FSE_2008_HGL_Temp_1}
\end{figure}

\begin{figure}[p]
\begin{tabular*}{\textwidth}{l@{\extracolsep{\fill}}r}
\includegraphics[height=2.15in]{SCRIPT_FIGURES/NIST_FSE_2008/ISOHeptD12_HGL_Temperature} &
\includegraphics[height=2.15in]{SCRIPT_FIGURES/NIST_FSE_2008/ISOHeptD12_HGL_Height} \\
\includegraphics[height=2.15in]{SCRIPT_FIGURES/NIST_FSE_2008/ISOHeptD13_HGL_Temperature} &
\includegraphics[height=2.15in]{SCRIPT_FIGURES/NIST_FSE_2008/ISOHeptD13_HGL_Height} \\
\includegraphics[height=2.15in]{SCRIPT_FIGURES/NIST_FSE_2008/ISOPropD14_HGL_Temperature} &
\includegraphics[height=2.15in]{SCRIPT_FIGURES/NIST_FSE_2008/ISOPropD14_HGL_Height} \\
\includegraphics[height=2.15in]{SCRIPT_FIGURES/NIST_FSE_2008/ISOProp15_HGL_Temperature} &
\includegraphics[height=2.15in]{SCRIPT_FIGURES/NIST_FSE_2008/ISOProp15_HGL_Height}
\end{tabular*}
\caption[NIST FSE, HGL temperature and height, Tests 12-15]
{NIST FSE, HGL temperature and height, Tests 12-15.}
\label{NIST_FSE_2008_HGL_Temp_2}
\end{figure}

\begin{figure}[p]
\begin{tabular*}{\textwidth}{l@{\extracolsep{\fill}}r}
\includegraphics[height=2.15in]{SCRIPT_FIGURES/NIST_FSE_2008/ISOStyrene16_HGL_Temperature} &
\includegraphics[height=2.15in]{SCRIPT_FIGURES/NIST_FSE_2008/ISOStyrene16_HGL_Height} \\
\includegraphics[height=2.15in]{SCRIPT_FIGURES/NIST_FSE_2008/ISOStyrene17_HGL_Temperature} &
\includegraphics[height=2.15in]{SCRIPT_FIGURES/NIST_FSE_2008/ISOStyrene17_HGL_Height} \\
\includegraphics[height=2.15in]{SCRIPT_FIGURES/NIST_FSE_2008/ISOPP18_HGL_Temperature} &
\includegraphics[height=2.15in]{SCRIPT_FIGURES/NIST_FSE_2008/ISOPP18_HGL_Height} \\
\includegraphics[height=2.15in]{SCRIPT_FIGURES/NIST_FSE_2008/ISOHept19_HGL_Temperature} &
\includegraphics[height=2.15in]{SCRIPT_FIGURES/NIST_FSE_2008/ISOHept19_HGL_Height}
\end{tabular*}
\caption[NIST FSE, HGL temperature and height, Tests 16-19]
{NIST FSE, HGL temperature and height, Tests 16-19.}
\label{NIST_FSE_2008_HGL_Temp_3}
\end{figure}

\begin{figure}[p]
\begin{tabular*}{\textwidth}{l@{\extracolsep{\fill}}r}
\includegraphics[height=2.15in]{SCRIPT_FIGURES/NIST_FSE_2008/ISOToluene20_HGL_Temperature} &
\includegraphics[height=2.15in]{SCRIPT_FIGURES/NIST_FSE_2008/ISOToluene20_HGL_Height} \\
\includegraphics[height=2.15in]{SCRIPT_FIGURES/NIST_FSE_2008/ISOStyrene21_HGL_Temperature} &
\includegraphics[height=2.15in]{SCRIPT_FIGURES/NIST_FSE_2008/ISOStyrene21_HGL_Height} \\
\includegraphics[height=2.15in]{SCRIPT_FIGURES/NIST_FSE_2008/ISOHept22_HGL_Temperature} &
\includegraphics[height=2.15in]{SCRIPT_FIGURES/NIST_FSE_2008/ISOHept22_HGL_Height} \\
\includegraphics[height=2.15in]{SCRIPT_FIGURES/NIST_FSE_2008/ISOHept23_HGL_Temperature} &
\includegraphics[height=2.15in]{SCRIPT_FIGURES/NIST_FSE_2008/ISOHept23_HGL_Height}
\end{tabular*}
\caption[NIST FSE, HGL temperature and height, Tests 20-23]
{NIST FSE, HGL temperature and height, Tests 20-23.}
\label{NIST_FSE_2008_HGL_Temp_4}
\end{figure}

\begin{figure}[p]
\begin{tabular*}{\textwidth}{l@{\extracolsep{\fill}}r}
\includegraphics[height=2.15in]{SCRIPT_FIGURES/NIST_FSE_2008/ISOHept24_HGL_Temperature} &
\includegraphics[height=2.15in]{SCRIPT_FIGURES/NIST_FSE_2008/ISOHept24_HGL_Height} \\
\includegraphics[height=2.15in]{SCRIPT_FIGURES/NIST_FSE_2008/ISOHept25_HGL_Temperature} &
\includegraphics[height=2.15in]{SCRIPT_FIGURES/NIST_FSE_2008/ISOHept25_HGL_Height} \\
\includegraphics[height=2.15in]{SCRIPT_FIGURES/NIST_FSE_2008/ISOHept26_HGL_Temperature} &
\includegraphics[height=2.15in]{SCRIPT_FIGURES/NIST_FSE_2008/ISOHept26_HGL_Height} \\
\includegraphics[height=2.15in]{SCRIPT_FIGURES/NIST_FSE_2008/ISOHept27_HGL_Temperature} &
\includegraphics[height=2.15in]{SCRIPT_FIGURES/NIST_FSE_2008/ISOHept27_HGL_Height}
\end{tabular*}
\caption[NIST FSE, HGL temperature and height, Tests 24-27]
{NIST FSE, HGL temperature and height, Tests 24-27.}
\label{NIST_FSE_2008_HGL_Temp_5}
\end{figure}

\begin{figure}[p]
\begin{tabular*}{\textwidth}{l@{\extracolsep{\fill}}r}
\includegraphics[height=2.15in]{SCRIPT_FIGURES/NIST_FSE_2008/ISOHept28_HGL_Temperature} &
\includegraphics[height=2.15in]{SCRIPT_FIGURES/NIST_FSE_2008/ISOHept28_HGL_Height} \\
\includegraphics[height=2.15in]{SCRIPT_FIGURES/NIST_FSE_2008/ISOToluene29_HGL_Temperature} &
\includegraphics[height=2.15in]{SCRIPT_FIGURES/NIST_FSE_2008/ISOToluene29_HGL_Height} \\
\includegraphics[height=2.15in]{SCRIPT_FIGURES/NIST_FSE_2008/ISOPropanol30_HGL_Temperature} &
\includegraphics[height=2.15in]{SCRIPT_FIGURES/NIST_FSE_2008/ISOPropanol30_HGL_Height} \\
\includegraphics[height=2.15in]{SCRIPT_FIGURES/NIST_FSE_2008/ISONG32_HGL_Temperature} &
\includegraphics[height=2.15in]{SCRIPT_FIGURES/NIST_FSE_2008/ISONG32_HGL_Height}
\end{tabular*}
\caption[NIST FSE, HGL temperature and height, Tests 28-30, 32]
{NIST FSE, HGL temperature and height, Tests 28-30,, 32.}
\label{NIST_FSE_2008_HGL_Temp_6}
\end{figure}


\clearpage

\section{NIST/NRC Test Series}

The NIST/NRC series consisted of 15 heptane spray fire experiments with varying heat release rates, pan locations, and ventilation conditions. Gas temperatures were measured using seven floor-to-ceiling thermocouple arrays (or ``trees'') distributed throughout the compartment.  The average hot gas layer temperature and height are calculated using thermocouple Trees 1, 2, 3, 5, 6 and 7. Tree~4 was not used because one of its thermocouples (TC~4-9) malfunctioned during most of the experiments. A few observations about the simulations:
\begin{itemize}
\item During Tests~4, 5, 10 and 16 a fan blew air into the compartment through a vent in the south wall. The measured velocity profile of the fan was not uniform, with the bulk of the air blowing from the lower third of the duct towards the ceiling at a roughly 45$^\circ$ angle.  The exact flow pattern is difficult to replicate in the model, thus, the results for Tests~4, 5, 10 and 16 should be evaluated with this in mind. The effect of the fan on the hot gas layer is small, but it does have a some effect on target temperatures near the vent.
\item For all of the tests involving a fan, the predicted HGL height increased after the fire was extinguished, while the measured HGL decreased.  This appears to be a curious artifact of the layer reduction algorithm. It is not included in the calculation of the relative difference.
\item In the closed door tests, the hot gas layer descended all the way to the floor. However, the reduction method, used on both the measured and predicted temperatures, does not account for the formation of a single layer, and therefore does not indicate that the layer drops all the way to the floor. This is neither a flaw in the measurements nor in FDS, but rather in the layer reduction method.
\item The HGL reduction method produces spurious results in the first few minutes of each test because no clear layer has yet formed. These early times are not included in the relative difference calculation.
\end{itemize}

\newpage

\begin{figure}[p]
\begin{tabular*}{\textwidth}{l@{\extracolsep{\fill}}r}
\includegraphics[height=2.15in]{SCRIPT_FIGURES/NIST_NRC/NIST_NRC_01_HGL_Temp} &
\includegraphics[height=2.15in]{SCRIPT_FIGURES/NIST_NRC/NIST_NRC_01_HGL_Height} \\
\includegraphics[height=2.15in]{SCRIPT_FIGURES/NIST_NRC/NIST_NRC_07_HGL_Temp} &
\includegraphics[height=2.15in]{SCRIPT_FIGURES/NIST_NRC/NIST_NRC_07_HGL_Height} \\
\includegraphics[height=2.15in]{SCRIPT_FIGURES/NIST_NRC/NIST_NRC_02_HGL_Temp} &
\includegraphics[height=2.15in]{SCRIPT_FIGURES/NIST_NRC/NIST_NRC_02_HGL_Height} \\
\includegraphics[height=2.15in]{SCRIPT_FIGURES/NIST_NRC/NIST_NRC_08_HGL_Temp} &
\includegraphics[height=2.15in]{SCRIPT_FIGURES/NIST_NRC/NIST_NRC_08_HGL_Height}
\end{tabular*}
\caption[NIST/NRC experiments, HGL temperature and height, Tests 1-2, 7-8]
{NIST/NRC experiments, HGL temperature and height, Tests 1-2, 7-8.}
\label{NIST_NRC_HGL_1}
\end{figure}

\begin{figure}[p]
\begin{tabular*}{\textwidth}{l@{\extracolsep{\fill}}r}
\includegraphics[height=2.15in]{SCRIPT_FIGURES/NIST_NRC/NIST_NRC_04_HGL_Temp} &
\includegraphics[height=2.15in]{SCRIPT_FIGURES/NIST_NRC/NIST_NRC_04_HGL_Height} \\
\includegraphics[height=2.15in]{SCRIPT_FIGURES/NIST_NRC/NIST_NRC_10_HGL_Temp} &
\includegraphics[height=2.15in]{SCRIPT_FIGURES/NIST_NRC/NIST_NRC_10_HGL_Height} \\
\includegraphics[height=2.15in]{SCRIPT_FIGURES/NIST_NRC/NIST_NRC_13_HGL_Temp} &
\includegraphics[height=2.15in]{SCRIPT_FIGURES/NIST_NRC/NIST_NRC_13_HGL_Height} \\
\includegraphics[height=2.15in]{SCRIPT_FIGURES/NIST_NRC/NIST_NRC_16_HGL_Temp} &
\includegraphics[height=2.15in]{SCRIPT_FIGURES/NIST_NRC/NIST_NRC_16_HGL_Height}
\end{tabular*}
\caption[NIST/NRC experiments, HGL temperature and height, Tests 4, 10, 13, 16]
{NIST/NRC experiments, HGL temperature and height, Tests 4, 10, 13, 16.}
\label{NIST_NRC_HGL_2}
\end{figure}

\begin{figure}[p]
\begin{tabular*}{\textwidth}{l@{\extracolsep{\fill}}r}
\includegraphics[height=2.15in]{SCRIPT_FIGURES/NIST_NRC/NIST_NRC_17_HGL_Temp} &
\includegraphics[height=2.15in]{SCRIPT_FIGURES/NIST_NRC/NIST_NRC_17_HGL_Height} \\ [1.in]
\multicolumn{2}{c}{Open door tests to follow} \\ [1.in]
\includegraphics[height=2.15in]{SCRIPT_FIGURES/NIST_NRC/NIST_NRC_03_HGL_Temp} &
\includegraphics[height=2.15in]{SCRIPT_FIGURES/NIST_NRC/NIST_NRC_03_HGL_Height} \\
\includegraphics[height=2.15in]{SCRIPT_FIGURES/NIST_NRC/NIST_NRC_09_HGL_Temp} &
\includegraphics[height=2.15in]{SCRIPT_FIGURES/NIST_NRC/NIST_NRC_09_HGL_Height}
\end{tabular*}
\caption[NIST/NRC experiments, HGL temperature and height, Tests 3, 9, 17]
{NIST/NRC experiments, HGL temperature and height, Tests 3, 9, 17.}
\label{NIST_NRC_HGL_3}
\end{figure}

\begin{figure}[p]
\begin{tabular*}{\textwidth}{l@{\extracolsep{\fill}}r}
\includegraphics[height=2.15in]{SCRIPT_FIGURES/NIST_NRC/NIST_NRC_05_HGL_Temp} &
\includegraphics[height=2.15in]{SCRIPT_FIGURES/NIST_NRC/NIST_NRC_05_HGL_Height} \\
\includegraphics[height=2.15in]{SCRIPT_FIGURES/NIST_NRC/NIST_NRC_14_HGL_Temp} &
\includegraphics[height=2.15in]{SCRIPT_FIGURES/NIST_NRC/NIST_NRC_14_HGL_Height} \\
\includegraphics[height=2.15in]{SCRIPT_FIGURES/NIST_NRC/NIST_NRC_15_HGL_Temp} &
\includegraphics[height=2.15in]{SCRIPT_FIGURES/NIST_NRC/NIST_NRC_15_HGL_Height} \\
\includegraphics[height=2.15in]{SCRIPT_FIGURES/NIST_NRC/NIST_NRC_18_HGL_Temp} &
\includegraphics[height=2.15in]{SCRIPT_FIGURES/NIST_NRC/NIST_NRC_18_HGL_Height}
\end{tabular*}
\caption[NIST/NRC experiments, HGL temperature and height, Tests 5, 14, 15, 18]
{NIST/NRC experiments, HGL temperature and height, Tests 5, 14, 15, 18.}
\label{NIST_NRC_HGL_4}
\end{figure}


\clearpage

\section{NIST/NRC Corner Effects Experiments}

The plots on the following pages compare hot gas layer temperatures and heights in a large compartment where corner, wall, and cabinet effects experiments were conducted. The corner and wall experiments involved a 60~cm by 60~cm natural gas burner with heat release rates of 200~kW, 300~kW, and 400~kW. The burner was either set in a corner or against a wall. The cabinet experiments involved gas burners set in one of two mock steel cabinets, with a variety of heat release rates.

In all experiments, two vertical thermocouple arrays were placed along the centerline of the room, each one-third of the room length from each respective short wall. The arrays each had 13 bare-bead thermocouples. The first was 2~cm below the ceiling, and the rest were spaced 30~cm apart.

\newpage

\begin{figure}[p]
\begin{tabular*}{\textwidth}{l@{\extracolsep{\fill}}r}
\includegraphics[height=2.15in]{SCRIPT_FIGURES/NIST_NRC_Corner_Effects/corner_200_kW_HGL_Temp} &
\includegraphics[height=2.15in]{SCRIPT_FIGURES/NIST_NRC_Corner_Effects/corner_200_kW_HGL_Height} \\
\includegraphics[height=2.15in]{SCRIPT_FIGURES/NIST_NRC_Corner_Effects/corner_300_kW_HGL_Temp} &
\includegraphics[height=2.15in]{SCRIPT_FIGURES/NIST_NRC_Corner_Effects/corner_300_kW_HGL_Height} \\
\includegraphics[height=2.15in]{SCRIPT_FIGURES/NIST_NRC_Corner_Effects/corner_400_kW_HGL_Temp} &
\includegraphics[height=2.15in]{SCRIPT_FIGURES/NIST_NRC_Corner_Effects/corner_400_kW_HGL_Height}
\end{tabular*}
\caption[NIST/NRC Corner Effects, HGL temperature and height, corner experiments]
{NIST/NRC Corner Effects experiments, HGL temperature and height, corner experiments.}
\label{NIST_NRC_Corner}
\end{figure}

\begin{figure}[p]
\begin{tabular*}{\textwidth}{l@{\extracolsep{\fill}}r}
\includegraphics[height=2.15in]{SCRIPT_FIGURES/NIST_NRC_Corner_Effects/wall_200_kW_HGL_Temp} &
\includegraphics[height=2.15in]{SCRIPT_FIGURES/NIST_NRC_Corner_Effects/wall_200_kW_HGL_Height} \\
\includegraphics[height=2.15in]{SCRIPT_FIGURES/NIST_NRC_Corner_Effects/wall_300_kW_HGL_Temp} &
\includegraphics[height=2.15in]{SCRIPT_FIGURES/NIST_NRC_Corner_Effects/wall_300_kW_HGL_Height} \\
\includegraphics[height=2.15in]{SCRIPT_FIGURES/NIST_NRC_Corner_Effects/wall_400_kW_HGL_Temp} &
\includegraphics[height=2.15in]{SCRIPT_FIGURES/NIST_NRC_Corner_Effects/wall_400_kW_HGL_Height}
\end{tabular*}
\caption[NIST/NRC Corner Effects, HGL temperature and height, wall experiments]
{NIST/NRC Corner Effects experiments, HGL temperature and height, wall experiments.}
\label{NIST_NRC_Wall}
\end{figure}

\begin{figure}[p]
\begin{tabular*}{\textwidth}{l@{\extracolsep{\fill}}r}
\includegraphics[height=2.15in]{SCRIPT_FIGURES/NIST_NRC_Corner_Effects/cabinet_01_HGL_Temp} &
\includegraphics[height=2.15in]{SCRIPT_FIGURES/NIST_NRC_Corner_Effects/cabinet_01_HGL_Height} \\
\includegraphics[height=2.15in]{SCRIPT_FIGURES/NIST_NRC_Corner_Effects/cabinet_02_HGL_Temp} &
\includegraphics[height=2.15in]{SCRIPT_FIGURES/NIST_NRC_Corner_Effects/cabinet_02_HGL_Height} \\
\includegraphics[height=2.15in]{SCRIPT_FIGURES/NIST_NRC_Corner_Effects/cabinet_03_HGL_Temp} &
\includegraphics[height=2.15in]{SCRIPT_FIGURES/NIST_NRC_Corner_Effects/cabinet_03_HGL_Height} \\
\includegraphics[height=2.15in]{SCRIPT_FIGURES/NIST_NRC_Corner_Effects/cabinet_04_HGL_Temp} &
\includegraphics[height=2.15in]{SCRIPT_FIGURES/NIST_NRC_Corner_Effects/cabinet_04_HGL_Height}
\end{tabular*}
\caption[NIST/NRC Corner Effects, HGL temperature and height, cabinet experiments 1-4]
{NIST/NRC Corner Effects experiments, HGL temperature and height, cabinet experiments 1-4.}
\label{NIST_NRC_Cabinet_1}
\end{figure}

\begin{figure}[p]
\begin{tabular*}{\textwidth}{l@{\extracolsep{\fill}}r}

\includegraphics[height=2.15in]{SCRIPT_FIGURES/NIST_NRC_Corner_Effects/cabinet_05_HGL_Temp} &
\includegraphics[height=2.15in]{SCRIPT_FIGURES/NIST_NRC_Corner_Effects/cabinet_05_HGL_Height} \\
\includegraphics[height=2.15in]{SCRIPT_FIGURES/NIST_NRC_Corner_Effects/cabinet_06_HGL_Temp} &
\includegraphics[height=2.15in]{SCRIPT_FIGURES/NIST_NRC_Corner_Effects/cabinet_06_HGL_Height} \\
\includegraphics[height=2.15in]{SCRIPT_FIGURES/NIST_NRC_Corner_Effects/cabinet_07_HGL_Temp} &
\includegraphics[height=2.15in]{SCRIPT_FIGURES/NIST_NRC_Corner_Effects/cabinet_07_HGL_Height} \\
\includegraphics[height=2.15in]{SCRIPT_FIGURES/NIST_NRC_Corner_Effects/cabinet_08_HGL_Temp} &
\includegraphics[height=2.15in]{SCRIPT_FIGURES/NIST_NRC_Corner_Effects/cabinet_08_HGL_Height}
\end{tabular*}
\caption[NIST/NRC Corner Effects, HGL temperature and height, cabinet experiments 5-8]
{NIST/NRC Corner Effects experiments, HGL temperature and height, cabinet experiments 5-8.}
\label{NIST_NRC_Cabinet_2}
\end{figure}

\begin{figure}[p]
\begin{tabular*}{\textwidth}{l@{\extracolsep{\fill}}r}
\includegraphics[height=2.15in]{SCRIPT_FIGURES/NIST_NRC_Corner_Effects/cabinet_09_HGL_Temp} &
\includegraphics[height=2.15in]{SCRIPT_FIGURES/NIST_NRC_Corner_Effects/cabinet_09_HGL_Height} \\
\includegraphics[height=2.15in]{SCRIPT_FIGURES/NIST_NRC_Corner_Effects/cabinet_10_HGL_Temp} &
\includegraphics[height=2.15in]{SCRIPT_FIGURES/NIST_NRC_Corner_Effects/cabinet_10_HGL_Height} \\
\includegraphics[height=2.15in]{SCRIPT_FIGURES/NIST_NRC_Corner_Effects/cabinet_11_HGL_Temp} &
\includegraphics[height=2.15in]{SCRIPT_FIGURES/NIST_NRC_Corner_Effects/cabinet_11_HGL_Height} \\
\includegraphics[height=2.15in]{SCRIPT_FIGURES/NIST_NRC_Corner_Effects/cabinet_12_HGL_Temp} &
\includegraphics[height=2.15in]{SCRIPT_FIGURES/NIST_NRC_Corner_Effects/cabinet_12_HGL_Height}
\end{tabular*}
\caption[NIST/NRC Corner Effects, HGL temperature and height, cabinet experiments 9-12]
{NIST/NRC Corner Effects experiments, HGL temperature and height, cabinet experiments 9-12.}
\label{NIST_NRC_Cabinet_3}
\end{figure}


\clearpage

\section{NIST Vent Study}

These experiments were performed in a small-scale two floor enclosure, with each floor connected by one or two ceiling vents. Each floor contained a vertical array of eight sheathed thermocouples at distances below the ceiling of 5, 10, 15, 20, 25, 30, 40, and 50~cm. Fifteen experiments were performed, but only 12 were modeled because three experiments involved a circular vent rather than square which could not be distinguished in the FDS simulations. The results of these experiments were nearly identical.

\newpage

\begin{figure}[p]
\begin{tabular*}{\textwidth}{l@{\extracolsep{\fill}}r}
\includegraphics[height=2.15in]{SCRIPT_FIGURES/NIST_Vent_Study/Test_01_HGL_Temp_Floor_1} &
\includegraphics[height=2.15in]{SCRIPT_FIGURES/NIST_Vent_Study/Test_01_HGL_Height_Floor_1} \\
\includegraphics[height=2.15in]{SCRIPT_FIGURES/NIST_Vent_Study/Test_01_HGL_Temp_Floor_2} &
\includegraphics[height=2.15in]{SCRIPT_FIGURES/NIST_Vent_Study/Test_01_HGL_Height_Floor_2} \\
\includegraphics[height=2.15in]{SCRIPT_FIGURES/NIST_Vent_Study/Test_02_HGL_Temp_Floor_1} &
\includegraphics[height=2.15in]{SCRIPT_FIGURES/NIST_Vent_Study/Test_02_HGL_Height_Floor_1} \\
\includegraphics[height=2.15in]{SCRIPT_FIGURES/NIST_Vent_Study/Test_02_HGL_Temp_Floor_2} &
\includegraphics[height=2.15in]{SCRIPT_FIGURES/NIST_Vent_Study/Test_02_HGL_Height_Floor_2}
\end{tabular*}
\caption[NIST Vent Study, HGL temperature and height, Tests 1 and 2]
{NIST Vent Study, HGL temperature and height, Tests 1 and 2.}
\label{NIST_Vent_Study_1_2}
\end{figure}

\begin{figure}[p]
\begin{tabular*}{\textwidth}{l@{\extracolsep{\fill}}r}
\includegraphics[height=2.15in]{SCRIPT_FIGURES/NIST_Vent_Study/Test_03_HGL_Temp_Floor_1} &
\includegraphics[height=2.15in]{SCRIPT_FIGURES/NIST_Vent_Study/Test_03_HGL_Height_Floor_1} \\
\includegraphics[height=2.15in]{SCRIPT_FIGURES/NIST_Vent_Study/Test_03_HGL_Temp_Floor_2} &
\includegraphics[height=2.15in]{SCRIPT_FIGURES/NIST_Vent_Study/Test_03_HGL_Height_Floor_2} \\
\includegraphics[height=2.15in]{SCRIPT_FIGURES/NIST_Vent_Study/Test_04_HGL_Temp_Floor_1} &
\includegraphics[height=2.15in]{SCRIPT_FIGURES/NIST_Vent_Study/Test_04_HGL_Height_Floor_1} \\
\includegraphics[height=2.15in]{SCRIPT_FIGURES/NIST_Vent_Study/Test_04_HGL_Temp_Floor_2} &
\includegraphics[height=2.15in]{SCRIPT_FIGURES/NIST_Vent_Study/Test_04_HGL_Height_Floor_2}
\end{tabular*}
\caption[NIST Vent Study, HGL temperature and height, Tests 3 and 4]
{NIST Vent Study, HGL temperature and height, Tests 3 and 4.}
\label{NIST_Vent_Study_3_4}
\end{figure}

\begin{figure}[p]
\begin{tabular*}{\textwidth}{l@{\extracolsep{\fill}}r}
\includegraphics[height=2.15in]{SCRIPT_FIGURES/NIST_Vent_Study/Test_05_HGL_Temp_Floor_1} &
\includegraphics[height=2.15in]{SCRIPT_FIGURES/NIST_Vent_Study/Test_05_HGL_Height_Floor_1} \\
\includegraphics[height=2.15in]{SCRIPT_FIGURES/NIST_Vent_Study/Test_05_HGL_Temp_Floor_2} &
\includegraphics[height=2.15in]{SCRIPT_FIGURES/NIST_Vent_Study/Test_05_HGL_Height_Floor_2} \\
\includegraphics[height=2.15in]{SCRIPT_FIGURES/NIST_Vent_Study/Test_06_HGL_Temp_Floor_1} &
\includegraphics[height=2.15in]{SCRIPT_FIGURES/NIST_Vent_Study/Test_06_HGL_Height_Floor_1} \\
\includegraphics[height=2.15in]{SCRIPT_FIGURES/NIST_Vent_Study/Test_06_HGL_Temp_Floor_2} &
\includegraphics[height=2.15in]{SCRIPT_FIGURES/NIST_Vent_Study/Test_06_HGL_Height_Floor_2}
\end{tabular*}
\caption[NIST Vent Study, HGL temperature and height, Tests 5 and 6]
{NIST Vent Study, HGL temperature and height, Tests 5 and 6.}
\label{NIST_Vent_Study_5_6}
\end{figure}

\begin{figure}[p]
\begin{tabular*}{\textwidth}{l@{\extracolsep{\fill}}r}
\includegraphics[height=2.15in]{SCRIPT_FIGURES/NIST_Vent_Study/Test_07_HGL_Temp_Floor_1} &
\includegraphics[height=2.15in]{SCRIPT_FIGURES/NIST_Vent_Study/Test_07_HGL_Height_Floor_1} \\
\includegraphics[height=2.15in]{SCRIPT_FIGURES/NIST_Vent_Study/Test_07_HGL_Temp_Floor_2} &
\includegraphics[height=2.15in]{SCRIPT_FIGURES/NIST_Vent_Study/Test_07_HGL_Height_Floor_2} \\
\includegraphics[height=2.15in]{SCRIPT_FIGURES/NIST_Vent_Study/Test_08_HGL_Temp_Floor_1} &
\includegraphics[height=2.15in]{SCRIPT_FIGURES/NIST_Vent_Study/Test_08_HGL_Height_Floor_1} \\
\includegraphics[height=2.15in]{SCRIPT_FIGURES/NIST_Vent_Study/Test_08_HGL_Temp_Floor_2} &
\includegraphics[height=2.15in]{SCRIPT_FIGURES/NIST_Vent_Study/Test_08_HGL_Height_Floor_2}
\end{tabular*}
\caption[NIST Vent Study, HGL temperature and height, Tests 7 and 8]
{NIST Vent Study, HGL temperature and height, Tests 7 and 8.}
\label{NIST_Vent_Study_7_8}
\end{figure}

\begin{figure}[p]
\begin{tabular*}{\textwidth}{l@{\extracolsep{\fill}}r}
\includegraphics[height=2.15in]{SCRIPT_FIGURES/NIST_Vent_Study/Test_09_HGL_Temp_Floor_1} &
\includegraphics[height=2.15in]{SCRIPT_FIGURES/NIST_Vent_Study/Test_09_HGL_Height_Floor_1} \\
\includegraphics[height=2.15in]{SCRIPT_FIGURES/NIST_Vent_Study/Test_09_HGL_Temp_Floor_2} &
\includegraphics[height=2.15in]{SCRIPT_FIGURES/NIST_Vent_Study/Test_09_HGL_Height_Floor_2} \\
\includegraphics[height=2.15in]{SCRIPT_FIGURES/NIST_Vent_Study/Test_13_HGL_Temp_Floor_1} &
\includegraphics[height=2.15in]{SCRIPT_FIGURES/NIST_Vent_Study/Test_13_HGL_Height_Floor_1} \\
\includegraphics[height=2.15in]{SCRIPT_FIGURES/NIST_Vent_Study/Test_13_HGL_Temp_Floor_2} &
\includegraphics[height=2.15in]{SCRIPT_FIGURES/NIST_Vent_Study/Test_13_HGL_Height_Floor_2}
\end{tabular*}
\caption[NIST Vent Study, HGL temperature and height, Tests 9 and 13]
{NIST Vent Study, HGL temperature and height, Tests 9 and 13.}
\label{NIST_Vent_Study_9_13}
\end{figure}

\begin{figure}[p]
\begin{tabular*}{\textwidth}{l@{\extracolsep{\fill}}r}
\includegraphics[height=2.15in]{SCRIPT_FIGURES/NIST_Vent_Study/Test_14_HGL_Temp_Floor_1} &
\includegraphics[height=2.15in]{SCRIPT_FIGURES/NIST_Vent_Study/Test_14_HGL_Height_Floor_1} \\
\includegraphics[height=2.15in]{SCRIPT_FIGURES/NIST_Vent_Study/Test_14_HGL_Temp_Floor_2} &
\includegraphics[height=2.15in]{SCRIPT_FIGURES/NIST_Vent_Study/Test_14_HGL_Height_Floor_2} \\
\includegraphics[height=2.15in]{SCRIPT_FIGURES/NIST_Vent_Study/Test_15_HGL_Temp_Floor_1} &
\includegraphics[height=2.15in]{SCRIPT_FIGURES/NIST_Vent_Study/Test_15_HGL_Height_Floor_1} \\
\includegraphics[height=2.15in]{SCRIPT_FIGURES/NIST_Vent_Study/Test_15_HGL_Temp_Floor_2} &
\includegraphics[height=2.15in]{SCRIPT_FIGURES/NIST_Vent_Study/Test_15_HGL_Height_Floor_2}
\end{tabular*}
\caption[NIST Vent Study, HGL temperature and height, Tests 14 and 15]
{NIST Vent Study, HGL temperature and height, Tests 14 and 15.}
\label{NIST_Vent_Study_14_15}
\end{figure}



\clearpage

\section{NRCC Smoke Tower}

In the NRCC Smoke Tower experiments, there was a vertical array consisting of thirteen TCs that were installed in the fire compartment on the second floor at the following heights: 0.62~m, 0.92~m, 1.22~m, 1.37~m, 1.52~m, 1.67~m, 1.82~m, 1.97~m, 2.12~m, 2.27~m, 2.42~m, 2.57~m and 2.95~m. Also, five TCs were installed in the doorway between the stair vestibule and stair shaft on the second floor. Figure~\ref{NRCC_Smoke_Tower_HGL_Temp} shows the predicted and measured HGL temperature for both vertical arrays.

\newpage

\begin{figure}[p]
\begin{tabular*}{\textwidth}{l@{\extracolsep{\fill}}r}
\includegraphics[height=2.15in]{SCRIPT_FIGURES/NRCC_Smoke_Tower/BK-R_Fire_Room_HGL_Temp} &
\includegraphics[height=2.15in]{SCRIPT_FIGURES/NRCC_Smoke_Tower/BK-R_Vestibule_HGL_Temp} \\
\includegraphics[height=2.15in]{SCRIPT_FIGURES/NRCC_Smoke_Tower/CMP-R_Fire_Room_HGL_Temp} &
\includegraphics[height=2.15in]{SCRIPT_FIGURES/NRCC_Smoke_Tower/CMP-R_Vestibule_HGL_Temp} \\
\includegraphics[height=2.15in]{SCRIPT_FIGURES/NRCC_Smoke_Tower/CLC-I-R_Fire_Room_HGL_Temp} &
\includegraphics[height=2.15in]{SCRIPT_FIGURES/NRCC_Smoke_Tower/CLC-I-R_Vestibule_HGL_Temp} \\
\includegraphics[height=2.15in]{SCRIPT_FIGURES/NRCC_Smoke_Tower/CLC-II-R_Fire_Room_HGL_Temp} &
\includegraphics[height=2.15in]{SCRIPT_FIGURES/NRCC_Smoke_Tower/CLC-II-R_Vestibule_HGL_Temp}
\end{tabular*}
\caption[NRCC Smoke Tower experiments, HGL temperature in the fire room and stair vestibule]
{NRCC Smoke Tower experiments, HGL temperature in the fire room and stair vestibule.}
\label{NRCC_Smoke_Tower_HGL_Temp}
\end{figure}


\clearpage

\section{PRISME DOOR Experiments}

The compartments in the PRISME DOOR experiments contained vertical arrays of thermocouples to measure the HGL temperature and depth. Each array contained 18 TCs and each compartment included three arrays. The array above the fire was excluded from the calculation of the HGL temperature and depth.

\begin{figure}[!ht]
\begin{tabular*}{\textwidth}{l@{\extracolsep{\fill}}r}
\includegraphics[height=2.15in]{SCRIPT_FIGURES/PRISME/PRS_D1_Room_1_HGL_Temp} &
\includegraphics[height=2.15in]{SCRIPT_FIGURES/PRISME/PRS_D1_Room_1_HGL_Height} \\
\includegraphics[height=2.15in]{SCRIPT_FIGURES/PRISME/PRS_D2_Room_1_HGL_Temp} &
\includegraphics[height=2.15in]{SCRIPT_FIGURES/PRISME/PRS_D2_Room_1_HGL_Height} \\
\includegraphics[height=2.15in]{SCRIPT_FIGURES/PRISME/PRS_D3_Room_1_HGL_Temp} &
\includegraphics[height=2.15in]{SCRIPT_FIGURES/PRISME/PRS_D3_Room_1_HGL_Height}
\end{tabular*}
\caption[PRISME DOOR experiments, HGL temperature and height, Room 1, Tests 1-3]
{PRISME DOOR experiments, HGL temperature and height, Room 1, Tests 1-3.}
\label{PRISME_HGL_1}
\end{figure}

\newpage

\begin{figure}[p]
\begin{tabular*}{\textwidth}{l@{\extracolsep{\fill}}r}
\includegraphics[height=2.15in]{SCRIPT_FIGURES/PRISME/PRS_D4_Room_1_HGL_Temp} &
\includegraphics[height=2.15in]{SCRIPT_FIGURES/PRISME/PRS_D4_Room_1_HGL_Height} \\
\includegraphics[height=2.15in]{SCRIPT_FIGURES/PRISME/PRS_D5_Room_1_HGL_Temp} &
\includegraphics[height=2.15in]{SCRIPT_FIGURES/PRISME/PRS_D5_Room_1_HGL_Height} \\
\includegraphics[height=2.15in]{SCRIPT_FIGURES/PRISME/PRS_D6_Room_1_HGL_Temp} &
\includegraphics[height=2.15in]{SCRIPT_FIGURES/PRISME/PRS_D6_Room_1_HGL_Height}
\end{tabular*}
\caption[PRISME DOOR experiments, HGL temperature and height, Room 1, Tests 4-6]
{PRISME DOOR experiments, HGL temperature and height, Room 1, Tests 4-6.}
\label{PRISME_HGL_2}
\end{figure}

\begin{figure}[p]
\begin{tabular*}{\textwidth}{l@{\extracolsep{\fill}}r}
\includegraphics[height=2.15in]{SCRIPT_FIGURES/PRISME/PRS_D1_Room_2_HGL_Temp} &
\includegraphics[height=2.15in]{SCRIPT_FIGURES/PRISME/PRS_D1_Room_2_HGL_Height} \\
\includegraphics[height=2.15in]{SCRIPT_FIGURES/PRISME/PRS_D2_Room_2_HGL_Temp} &
\includegraphics[height=2.15in]{SCRIPT_FIGURES/PRISME/PRS_D2_Room_2_HGL_Height} \\
\includegraphics[height=2.15in]{SCRIPT_FIGURES/PRISME/PRS_D3_Room_2_HGL_Temp} &
\includegraphics[height=2.15in]{SCRIPT_FIGURES/PRISME/PRS_D3_Room_2_HGL_Height}
\end{tabular*}
\caption[PRISME DOOR experiments, HGL temperature and height, Room 2, Tests 1-3]
{PRISME DOOR experiments, HGL temperature and height, Room 2, Tests 1-3.}
\label{PRISME_HGL_3}
\end{figure}

\begin{figure}[p]
\begin{tabular*}{\textwidth}{l@{\extracolsep{\fill}}r}
\includegraphics[height=2.15in]{SCRIPT_FIGURES/PRISME/PRS_D4_Room_2_HGL_Temp} &
\includegraphics[height=2.15in]{SCRIPT_FIGURES/PRISME/PRS_D4_Room_2_HGL_Height} \\
\includegraphics[height=2.15in]{SCRIPT_FIGURES/PRISME/PRS_D5_Room_2_HGL_Temp} &
\includegraphics[height=2.15in]{SCRIPT_FIGURES/PRISME/PRS_D5_Room_2_HGL_Height} \\
\includegraphics[height=2.15in]{SCRIPT_FIGURES/PRISME/PRS_D6_Room_2_HGL_Temp} &
\includegraphics[height=2.15in]{SCRIPT_FIGURES/PRISME/PRS_D6_Room_2_HGL_Height}
\end{tabular*}
\caption[PRISME DOOR experiments, HGL temperature and height, Room 2, Tests 4-6]
{PRISME DOOR experiments, HGL temperature and height, Room 1, Tests 4-6.}
\label{PRISME_HGL_4}
\end{figure}


\clearpage

\section{PRISME SOURCE Experiments}

The PRISME SOURCE experiments were conducted in a single compartment connected to an HVAC network. The compartment was 5~m by 6~m by 4~m high. The HGL temperature was computed from a single vertical thermocouple array located in the northeast quadrant of the compartment. The array contained 18 TCs; the highest one 0.1~m below the ceiling.

\newpage

\begin{figure}[p]
\begin{tabular*}{\textwidth}{l@{\extracolsep{\fill}}r}
\includegraphics[height=2.15in]{SCRIPT_FIGURES/PRISME/PRS_SI_D1_Room_2_HGL_Temp} &
\includegraphics[height=2.15in]{SCRIPT_FIGURES/PRISME/PRS_SI_D1_Room_2_HGL_Height} \\
\includegraphics[height=2.15in]{SCRIPT_FIGURES/PRISME/PRS_SI_D2_Room_2_HGL_Temp} &
\includegraphics[height=2.15in]{SCRIPT_FIGURES/PRISME/PRS_SI_D2_Room_2_HGL_Height} \\
\includegraphics[height=2.15in]{SCRIPT_FIGURES/PRISME/PRS_SI_D3_Room_2_HGL_Temp} &
\includegraphics[height=2.15in]{SCRIPT_FIGURES/PRISME/PRS_SI_D3_Room_2_HGL_Height} \\
\includegraphics[height=2.15in]{SCRIPT_FIGURES/PRISME/PRS_SI_D4_Room_2_HGL_Temp} &
\includegraphics[height=2.15in]{SCRIPT_FIGURES/PRISME/PRS_SI_D4_Room_2_HGL_Height}
\end{tabular*}
\caption[PRISME SOURCE experiments, HGL temperature and height, Room 2, Tests 1-4]
{PRISME SOURCE experiments, HGL temperature and height, Room 2, Tests 1-4.}
\label{PRISME_SI_HGL_1}
\end{figure}

\begin{figure}[p]
\begin{tabular*}{\textwidth}{l@{\extracolsep{\fill}}r}
\includegraphics[height=2.15in]{SCRIPT_FIGURES/PRISME/PRS_SI_D5_Room_2_HGL_Temp} &
\includegraphics[height=2.15in]{SCRIPT_FIGURES/PRISME/PRS_SI_D5_Room_2_HGL_Height} \\
\includegraphics[height=2.15in]{SCRIPT_FIGURES/PRISME/PRS_SI_D5a_Room_2_HGL_Temp} &
\includegraphics[height=2.15in]{SCRIPT_FIGURES/PRISME/PRS_SI_D5a_Room_2_HGL_Height} \\
\includegraphics[height=2.15in]{SCRIPT_FIGURES/PRISME/PRS_SI_D6_Room_2_HGL_Temp} &
\includegraphics[height=2.15in]{SCRIPT_FIGURES/PRISME/PRS_SI_D6_Room_2_HGL_Height} \\
\includegraphics[height=2.15in]{SCRIPT_FIGURES/PRISME/PRS_SI_D6a_Room_2_HGL_Temp} &
\includegraphics[height=2.15in]{SCRIPT_FIGURES/PRISME/PRS_SI_D6a_Room_2_HGL_Height}
\end{tabular*}
\caption[PRISME SOURCE experiments, HGL temperature and height, Room 2, Tests 5-6]
{PRISME SOURCE experiments, HGL temperature and height, Room 2, Tests 5-6.}
\label{PRISME_SI_HGL_2}
\end{figure}

\clearpage

\section{Steckler Compartment Experiments}

Steckler et al.~\cite{Steckler:NBSIR_82-2520} mapped the doorway/window flows in 55 compartment fire experiments. The test matrix is presented in Table~\ref{Steckler_Table}. Shown on the following pages are the temperature profiles inside the compartment compared with model predictions. To quantify the difference between prediction and measurement, the maximum temperatures were compared.

\newpage

\begin{figure}[p]
\begin{tabular*}{\textwidth}{l@{\extracolsep{\fill}}r}
\includegraphics[height=2.15in]{SCRIPT_FIGURES/Steckler_Compartment/Steckler_010_Temp} &
\includegraphics[height=2.15in]{SCRIPT_FIGURES/Steckler_Compartment/Steckler_011_Temp} \\
\includegraphics[height=2.15in]{SCRIPT_FIGURES/Steckler_Compartment/Steckler_012_Temp} &
\includegraphics[height=2.15in]{SCRIPT_FIGURES/Steckler_Compartment/Steckler_612_Temp} \\
\includegraphics[height=2.15in]{SCRIPT_FIGURES/Steckler_Compartment/Steckler_013_Temp} &
\includegraphics[height=2.15in]{SCRIPT_FIGURES/Steckler_Compartment/Steckler_014_Temp} \\
\includegraphics[height=2.15in]{SCRIPT_FIGURES/Steckler_Compartment/Steckler_018_Temp} &
\includegraphics[height=2.15in]{SCRIPT_FIGURES/Steckler_Compartment/Steckler_710_Temp}
\end{tabular*}
\caption[Steckler experiments, HGL temperature, Tests 10, 11, 12, 13, 14, 18, 612, 710]
{Steckler experiments, HGL temperature, Tests 10, 11, 12, 13, 14, 18, 612, 710.}
\label{Steckler_Temp_1}
\end{figure}

\begin{figure}[p]
\begin{tabular*}{\textwidth}{l@{\extracolsep{\fill}}r}
\includegraphics[height=2.15in]{SCRIPT_FIGURES/Steckler_Compartment/Steckler_810_Temp} &
\includegraphics[height=2.15in]{SCRIPT_FIGURES/Steckler_Compartment/Steckler_016_Temp} \\
\includegraphics[height=2.15in]{SCRIPT_FIGURES/Steckler_Compartment/Steckler_017_Temp} &
\includegraphics[height=2.15in]{SCRIPT_FIGURES/Steckler_Compartment/Steckler_022_Temp} \\
\includegraphics[height=2.15in]{SCRIPT_FIGURES/Steckler_Compartment/Steckler_023_Temp} &
\includegraphics[height=2.15in]{SCRIPT_FIGURES/Steckler_Compartment/Steckler_030_Temp} \\
\includegraphics[height=2.15in]{SCRIPT_FIGURES/Steckler_Compartment/Steckler_041_Temp} &
\includegraphics[height=2.15in]{SCRIPT_FIGURES/Steckler_Compartment/Steckler_019_Temp}
\end{tabular*}
\caption[Steckler experiments, HGL temperature, Tests 16, 17, 19, 22, 23, 30, 41, 810]
{Steckler experiments, HGL temperature, Tests 16, 17, 19, 22, 23, 30, 41, 810.}
\label{Steckler_Temp_2}
\end{figure}

\begin{figure}[p]
\begin{tabular*}{\textwidth}{l@{\extracolsep{\fill}}r}
\includegraphics[height=2.15in]{SCRIPT_FIGURES/Steckler_Compartment/Steckler_020_Temp} &
\includegraphics[height=2.15in]{SCRIPT_FIGURES/Steckler_Compartment/Steckler_021_Temp} \\
\includegraphics[height=2.15in]{SCRIPT_FIGURES/Steckler_Compartment/Steckler_114_Temp} &
\includegraphics[height=2.15in]{SCRIPT_FIGURES/Steckler_Compartment/Steckler_144_Temp} \\
\includegraphics[height=2.15in]{SCRIPT_FIGURES/Steckler_Compartment/Steckler_212_Temp} &
\includegraphics[height=2.15in]{SCRIPT_FIGURES/Steckler_Compartment/Steckler_242_Temp} \\
\includegraphics[height=2.15in]{SCRIPT_FIGURES/Steckler_Compartment/Steckler_410_Temp} &
\includegraphics[height=2.15in]{SCRIPT_FIGURES/Steckler_Compartment/Steckler_210_Temp}
\end{tabular*}
\caption[Steckler experiments, HGL temperature, Tests 20, 21, 114, 144, 210, 212, 242, 410]
{Steckler experiments, HGL temperature, Tests 20, 21, 114, 144, 210, 212, 242, 410.}
\label{Steckler_Temp_3}
\end{figure}

\begin{figure}[p]
\begin{tabular*}{\textwidth}{l@{\extracolsep{\fill}}r}
\includegraphics[height=2.15in]{SCRIPT_FIGURES/Steckler_Compartment/Steckler_310_Temp} &
\includegraphics[height=2.15in]{SCRIPT_FIGURES/Steckler_Compartment/Steckler_240_Temp} \\
\includegraphics[height=2.15in]{SCRIPT_FIGURES/Steckler_Compartment/Steckler_116_Temp} &
\includegraphics[height=2.15in]{SCRIPT_FIGURES/Steckler_Compartment/Steckler_122_Temp} \\
\includegraphics[height=2.15in]{SCRIPT_FIGURES/Steckler_Compartment/Steckler_224_Temp} &
\includegraphics[height=2.15in]{SCRIPT_FIGURES/Steckler_Compartment/Steckler_324_Temp} \\
\includegraphics[height=2.15in]{SCRIPT_FIGURES/Steckler_Compartment/Steckler_220_Temp} &
\includegraphics[height=2.15in]{SCRIPT_FIGURES/Steckler_Compartment/Steckler_221_Temp}
\end{tabular*}
\caption[Steckler experiments, HGL temperature, Tests 116, 122, 220, 221, 224, 240,310, 324]
{Steckler experiments, HGL temperature, Tests 116, 122, 220, 221, 224, 240,310, 324.}
\label{Steckler_Temp_4}
\end{figure}

\begin{figure}[p]
\begin{tabular*}{\textwidth}{l@{\extracolsep{\fill}}r}
\includegraphics[height=2.15in]{SCRIPT_FIGURES/Steckler_Compartment/Steckler_514_Temp} &
\includegraphics[height=2.15in]{SCRIPT_FIGURES/Steckler_Compartment/Steckler_544_Temp} \\
\includegraphics[height=2.15in]{SCRIPT_FIGURES/Steckler_Compartment/Steckler_512_Temp} &
\includegraphics[height=2.15in]{SCRIPT_FIGURES/Steckler_Compartment/Steckler_542_Temp} \\
\includegraphics[height=2.15in]{SCRIPT_FIGURES/Steckler_Compartment/Steckler_610_Temp} &
\includegraphics[height=2.15in]{SCRIPT_FIGURES/Steckler_Compartment/Steckler_510_Temp} \\
\includegraphics[height=2.15in]{SCRIPT_FIGURES/Steckler_Compartment/Steckler_540_Temp} &
\includegraphics[height=2.15in]{SCRIPT_FIGURES/Steckler_Compartment/Steckler_517_Temp}
\end{tabular*}
\caption[Steckler experiments, HGL temperature, Tests 510, 512, 514, 517, 540, 542, 544, 610]
{Steckler experiments, HGL temperature, Tests 510, 512, 514, 517, 540, 542, 544, 610.}
\label{Steckler_Temp_5}
\end{figure}

\begin{figure}[p]
\begin{tabular*}{\textwidth}{l@{\extracolsep{\fill}}r}
\includegraphics[height=2.15in]{SCRIPT_FIGURES/Steckler_Compartment/Steckler_622_Temp} &
\includegraphics[height=2.15in]{SCRIPT_FIGURES/Steckler_Compartment/Steckler_522_Temp} \\
\includegraphics[height=2.15in]{SCRIPT_FIGURES/Steckler_Compartment/Steckler_524_Temp} &
\includegraphics[height=2.15in]{SCRIPT_FIGURES/Steckler_Compartment/Steckler_541_Temp} \\
\includegraphics[height=2.15in]{SCRIPT_FIGURES/Steckler_Compartment/Steckler_520_Temp} &
\includegraphics[height=2.15in]{SCRIPT_FIGURES/Steckler_Compartment/Steckler_521_Temp} \\
\includegraphics[height=2.15in]{SCRIPT_FIGURES/Steckler_Compartment/Steckler_513_Temp} &
\includegraphics[height=2.15in]{SCRIPT_FIGURES/Steckler_Compartment/Steckler_160_Temp}
\end{tabular*}
\caption[Steckler experiments, HGL temperature, Tests 160, 513, 520, 521, 522, 524, 541, 622]
{Steckler experiments, HGL temperature, Tests 160, 513, 520, 521, 522, 524, 541, 622.}
\label{Steckler_Temp_6}
\end{figure}

\begin{figure}[p]
\begin{tabular*}{\textwidth}{l@{\extracolsep{\fill}}r}
\includegraphics[height=2.15in]{SCRIPT_FIGURES/Steckler_Compartment/Steckler_163_Temp} &
\includegraphics[height=2.15in]{SCRIPT_FIGURES/Steckler_Compartment/Steckler_164_Temp} \\
\includegraphics[height=2.15in]{SCRIPT_FIGURES/Steckler_Compartment/Steckler_165_Temp} &
\includegraphics[height=2.15in]{SCRIPT_FIGURES/Steckler_Compartment/Steckler_162_Temp} \\
\includegraphics[height=2.15in]{SCRIPT_FIGURES/Steckler_Compartment/Steckler_167_Temp} &
\includegraphics[height=2.15in]{SCRIPT_FIGURES/Steckler_Compartment/Steckler_161_Temp} \\
\includegraphics[height=2.15in]{SCRIPT_FIGURES/Steckler_Compartment/Steckler_166_Temp} &
\end{tabular*}
\caption[Steckler experiments, HGL temperature, Tests 161, 162, 163, 164, 165, 166, 167]
{Steckler experiments, HGL temperature, Tests 161, 162, 163, 164, 165, 166, 167.}
\label{Steckler_Temp_7}
\end{figure}


\clearpage

\section{TUS Facade Experiments}

The TUS Facade experiments involved a 1.7~m tall burner compartment out of which spilled a fire plume through a rectangular opening. Shown in Fig.~\ref{TUS_Facade_HGL_Temp} are the HGL temperatures inside the compartment where the gas burners are located.

\newpage

\begin{figure}[p]
\begin{tabular*}{\textwidth}{l@{\extracolsep{\fill}}r}
\includegraphics[height=2.15in]{SCRIPT_FIGURES/TUS_Facade/HGL_Temperature_I_1} &
\includegraphics[height=2.15in]{SCRIPT_FIGURES/TUS_Facade/HGL_Temperature_I_2} \\
\includegraphics[height=2.15in]{SCRIPT_FIGURES/TUS_Facade/HGL_Temperature_1_1} &
\includegraphics[height=2.15in]{SCRIPT_FIGURES/TUS_Facade/HGL_Temperature_1_2} \\
\includegraphics[height=2.15in]{SCRIPT_FIGURES/TUS_Facade/HGL_Temperature_1_3} &
\includegraphics[height=2.15in]{SCRIPT_FIGURES/TUS_Facade/HGL_Temperature_2_1} \\
\includegraphics[height=2.15in]{SCRIPT_FIGURES/TUS_Facade/HGL_Temperature_2_2} &
\includegraphics[height=2.15in]{SCRIPT_FIGURES/TUS_Facade/HGL_Temperature_2_3}
\end{tabular*}
\caption[TUS Facade Experiments, HGL Temperature]
{TUS Facade Experiments, HGL Temperature.}
\label{TUS_Facade_HGL_Temp}
\end{figure}


\clearpage


\section{UL/NIST Vent Experiments}

The HGL temperature and height for the four experiments was calculated from two vertical arrays of eight thermocouples each. The arrays were centered on the long central axis of the compartment and 90~cm from each short size wall. The 2.4~m by 1.2~m double vent was 90~cm from each array. The uppermost TC was 2.5~cm below the ceiling. The second TC was 30~cm (1~ft) below the ceiling, and the rest were spaced evenly by 1~ft.

\newpage

\begin{figure}[p]
\begin{tabular*}{\textwidth}{l@{\extracolsep{\fill}}r}
\includegraphics[height=2.15in]{SCRIPT_FIGURES/UL_NIST_Vents/UL_NIST_Vents_Test_1_HGL_Temp} &
\includegraphics[height=2.15in]{SCRIPT_FIGURES/UL_NIST_Vents/UL_NIST_Vents_Test_1_HGL_Height} \\
\includegraphics[height=2.15in]{SCRIPT_FIGURES/UL_NIST_Vents/UL_NIST_Vents_Test_2_HGL_Temp} &
\includegraphics[height=2.15in]{SCRIPT_FIGURES/UL_NIST_Vents/UL_NIST_Vents_Test_2_HGL_Height} \\
\includegraphics[height=2.15in]{SCRIPT_FIGURES/UL_NIST_Vents/UL_NIST_Vents_Test_3_HGL_Temp} &
\includegraphics[height=2.15in]{SCRIPT_FIGURES/UL_NIST_Vents/UL_NIST_Vents_Test_3_HGL_Height} \\
\includegraphics[height=2.15in]{SCRIPT_FIGURES/UL_NIST_Vents/UL_NIST_Vents_Test_4_HGL_Temp} &
\includegraphics[height=2.15in]{SCRIPT_FIGURES/UL_NIST_Vents/UL_NIST_Vents_Test_4_HGL_Height}
\end{tabular*}
\caption[UL/NIST experiments, HGL temperature and height, Tests 1-4]
{UL/NIST experiments, HGL temperature and height, Tests 1-4.}
\label{UL_NIST_HGL}
\end{figure}

\clearpage

\section{UL/NIJ House Experiments}

Details of the experiments can be found in Sec.~\ref{UL_NIJ_Description}

The HGL temperature and height for three experiments conducted in the ranch-style house were calculated from two vertical arrays of eight thermocouples. The thermocouple arrays were located in the hallway (4TC) and in the living room (5TC).

The HGL temperature and height for three experiments conducted in the two-story colonial-style house were calculated from a vertical array of thermocouples in the center of the family room (8TC). The uppermost TC was 2.5~cm below the ceiling. The second TC was 30~cm (1~ft) below the ceiling, and the rest were spaced evenly by 30~cm (1~ft).

\newpage

\begin{figure}[p]
\begin{tabular*}{\textwidth}{l@{\extracolsep{\fill}}r}
\includegraphics[height=2.15in]{SCRIPT_FIGURES/UL_NIJ_Houses/UL_NIJ_Single_Story_Gas_1_HGL_Temp} &
\includegraphics[height=2.15in]{SCRIPT_FIGURES/UL_NIJ_Houses/UL_NIJ_Single_Story_Gas_1_HGL_Height} \\
\includegraphics[height=2.15in]{SCRIPT_FIGURES/UL_NIJ_Houses/UL_NIJ_Single_Story_Gas_2_HGL_Temp} &
\includegraphics[height=2.15in]{SCRIPT_FIGURES/UL_NIJ_Houses/UL_NIJ_Single_Story_Gas_2_HGL_Height} \\
\includegraphics[height=2.15in]{SCRIPT_FIGURES/UL_NIJ_Houses/UL_NIJ_Single_Story_Gas_5_HGL_Temp} &
\includegraphics[height=2.15in]{SCRIPT_FIGURES/UL_NIJ_Houses/UL_NIJ_Single_Story_Gas_5_HGL_Height} \\
\end{tabular*}
\caption[UL/NIJ Experiments, HGL temperature and height, Tests 1, 2, and 5]{UL/NIJ Experiments, HGL temperature and height, single-story ranch-style House Tests 1, 2, and 5}
\label{UL_NIJ_HGL_1}
\end{figure}

\begin{figure}[p]
\begin{tabular*}{\textwidth}{l@{\extracolsep{\fill}}r}
\includegraphics[height=2.15in]{SCRIPT_FIGURES/UL_NIJ_Houses/UL_NIJ_Two_Story_Gas_1_HGL_Temp} &
\includegraphics[height=2.15in]{SCRIPT_FIGURES/UL_NIJ_Houses/UL_NIJ_Two_Story_Gas_1_HGL_Height} \\
\includegraphics[height=2.15in]{SCRIPT_FIGURES/UL_NIJ_Houses/UL_NIJ_Two_Story_Gas_4_HGL_Temp} &
\includegraphics[height=2.15in]{SCRIPT_FIGURES/UL_NIJ_Houses/UL_NIJ_Two_Story_Gas_4_HGL_Height} \\
\includegraphics[height=2.15in]{SCRIPT_FIGURES/UL_NIJ_Houses/UL_NIJ_Two_Story_Gas_6_HGL_Temp} &
\includegraphics[height=2.15in]{SCRIPT_FIGURES/UL_NIJ_Houses/UL_NIJ_Two_Story_Gas_6_HGL_Height} \\
\end{tabular*}
\caption[UL/NIJ Experiments, HGL temperature and height, Tests 1, 4, and 6]{UL/NIJ Experiments, HGL temperature and height, two-story colonial-style House Tests 1, 4, and 6}
\label{UL_NIJ_HGL_2}
\end{figure}

\clearpage

\section{VTT Test Series}

The HGL temperature and height are calculated from the (1~min) averaged gas temperatures from three vertical thermocouple arrays using the standard reduction method. There are 10 thermocouples in each vertical array, spaced 2~m apart in the lower two-thirds of the hall, and 1~m apart near the ceiling.

\begin{figure}[h!]
\begin{tabular*}{\textwidth}{l@{\extracolsep{\fill}}r}
\includegraphics[height=2.15in]{SCRIPT_FIGURES/VTT/VTT_01_HGL_Temp} &
\includegraphics[height=2.15in]{SCRIPT_FIGURES/VTT/VTT_01_HGL_Height} \\
\includegraphics[height=2.15in]{SCRIPT_FIGURES/VTT/VTT_02_HGL_Temp} &
\includegraphics[height=2.15in]{SCRIPT_FIGURES/VTT/VTT_02_HGL_Height} \\
\includegraphics[height=2.15in]{SCRIPT_FIGURES/VTT/VTT_03_HGL_Temp} &
\includegraphics[height=2.15in]{SCRIPT_FIGURES/VTT/VTT_03_HGL_Height}
\end{tabular*}
\caption[VTT experiments, HGL temperature and height, Tests 1-3]
{VTT experiments, HGL temperature and height, Tests 1-3.}
\label{VTT_HGL}
\end{figure}



\clearpage

\section{WTC Test Series}

The HGL temperature and height for the WTC experiments were calculated from two TC trees, one that was approximately 3~m to the west and one
2~m to the east of the fire pan (see Fig.~\ref{WTC_Drawing}). Each tree consisted of 15 thermocouples, the highest point being 5~cm below the ceiling.

\begin{figure}[!h]
\begin{tabular*}{\textwidth}{l@{\extracolsep{\fill}}r}
\includegraphics[height=2.15in]{SCRIPT_FIGURES/WTC/WTC_01_HGL_Temp} &
\includegraphics[height=2.15in]{SCRIPT_FIGURES/WTC/WTC_01_HGL_Height} \\
\includegraphics[height=2.15in]{SCRIPT_FIGURES/WTC/WTC_02_HGL_Temp} &
\includegraphics[height=2.15in]{SCRIPT_FIGURES/WTC/WTC_02_HGL_Height} \\
\includegraphics[height=2.15in]{SCRIPT_FIGURES/WTC/WTC_03_HGL_Temp} &
\includegraphics[height=2.15in]{SCRIPT_FIGURES/WTC/WTC_03_HGL_Height}
\end{tabular*}
\caption[WTC experiments, HGL temperature and height, Tests 1-3]
{WTC experiments, HGL temperature and height, Tests 1-3.}
\label{WTC_HGL_1}
\end{figure}

\newpage

\begin{figure}[p]
\begin{tabular*}{\textwidth}{l@{\extracolsep{\fill}}r}
\includegraphics[height=2.15in]{SCRIPT_FIGURES/WTC/WTC_04_HGL_Temp} &
\includegraphics[height=2.15in]{SCRIPT_FIGURES/WTC/WTC_04_HGL_Height} \\
\includegraphics[height=2.15in]{SCRIPT_FIGURES/WTC/WTC_05_HGL_Temp} &
\includegraphics[height=2.15in]{SCRIPT_FIGURES/WTC/WTC_05_HGL_Height} \\
\includegraphics[height=2.15in]{SCRIPT_FIGURES/WTC/WTC_06_HGL_Temp} &
\includegraphics[height=2.15in]{SCRIPT_FIGURES/WTC/WTC_06_HGL_Height}
\end{tabular*}
\caption[WTC experiments, HGL temperature and height, Tests 4-6]
{WTC experiments, HGL temperature and height, Tests 4-6.}
\label{WTC_HGL_2}
\end{figure}


\clearpage

\section{Summary of Hot Gas Layer Temperature and Height}
\label{HGL Temperature, Natural Ventilation}
\label{HGL Temperature, Forced Ventilation}
\label{HGL Temperature, No Ventilation}
\label{HGL Depth}


\begin{figure}[!h]
\centering
\begin{tabular}{l}
\includegraphics[height=4in]{SCRIPT_FIGURES/ScatterPlots/FDS_HGL_Temperature_Natural_Ventilation} \\
\includegraphics[height=4in]{SCRIPT_FIGURES/ScatterPlots/FDS_HGL_Temperature_Forced_Ventilation}
\end{tabular}
\caption[Summary, HGL temperature, natural and forced ventilation]
{Summary, HGL temperature, natural and forced ventilation.}
\label{HGL_Summary_1}
\end{figure}

\newpage

\begin{figure}[!h]
\centering
\begin{tabular}{l}
\includegraphics[height=4in]{SCRIPT_FIGURES/ScatterPlots/FDS_HGL_Temperature_No_Ventilation} \\
\includegraphics[height=4in]{SCRIPT_FIGURES/ScatterPlots/FDS_HGL_Depth}
\end{tabular}
\caption[Summary, HGL temperature, unventilated compartments; HGL depth]
{Summary, HGL temperature, unventilated compartments; HGL depth.}
\label{HGL_Summary_2}
\end{figure}



